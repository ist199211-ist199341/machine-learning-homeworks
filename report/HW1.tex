\documentclass[12pt]{article}
\usepackage[paper=letterpaper,margin=2cm]{geometry}
\usepackage{amsmath}
\usepackage{amssymb}
\usepackage{amsfonts}
\usepackage{newtxtext, newtxmath}
\usepackage{enumitem}
\usepackage{titling}
\usepackage[colorlinks=true]{hyperref}
\usepackage{multirow}
\usepackage{svg}
\usepackage{listings}
\usepackage{xcolor}

\setlength{\droptitle}{-6em}

\definecolor{codegreen}{rgb}{0,0.6,0}
\definecolor{codegray}{rgb}{0.5,0.5,0.5}
\definecolor{codepurple}{rgb}{0.58,0,0.82}
\definecolor{backcolour}{rgb}{0.95,0.95,0.92}

\lstdefinestyle{mystyle}{
    commentstyle=\color{codegreen},
    keywordstyle=\color{magenta},
    numberstyle=\tiny\color{codegray},
    stringstyle=\color{codepurple},
    basicstyle=\ttfamily\footnotesize,
    breakatwhitespace=false,
    breaklines=true,
    captionpos=b,
    keepspaces=true,
    numbers=left,
    numbersep=5pt,
    showspaces=false,
    showstringspaces=false,
    showtabs=false,
    tabsize=2
}

\lstset{style=mystyle}

\title{\large{Aprendizagem 2022}\vskip 0.2cm Homework I -- Group 020}
\date{}
\author{Diogo Correia (99211) \and Tomás Esteves (99341)}
\begin{document}
\maketitle
\center\large{\vskip -1.0cm\textbf{Part I}: Pen and paper}
\begin{enumerate}[leftmargin=\labelsep]
    Given the following decision tree learnt from 20 observation using Shannon entropy, with leaf annotations (\texttt{\#correct/\#total}).

    \begin{figure}[h!]
        \centering
        \includesvg[width=7cm]{assets/hw1-tree1.svg}
        \caption{Decision Tree from Part I}
        \label{fig:decision-tree}
    \end{figure}

    \item \textbf{Draw the training confusion matrix.}\\
          \vspace{0.5em}
          \begin{center}

              \begin{tabular}{|c|c|c|c|c|}
                  \cline{3-4}
                  \multicolumn{2}{c}{}                & \multicolumn{2}{|c|}{\textbf{Actual}} & \multicolumn{1}{c}{}                             \\
                  \cline{3-4}
                  \multicolumn{2}{c|}{}               & \textbf{Positive}                     & \textbf{Negative}    & \multicolumn{1}{c}{}      \\
                  \hline
                  \multirow{2}{*}{\textbf{Predicted}} & \textbf{Positive}                     & 8                    & 4                    & 12 \\
                  \cline{2-5}
                                                      & \textbf{Negative}                     & 3                    & 5                    & 8  \\
                  \hline
                  \multicolumn{2}{c|}{}               & 11                                    & 9                    & 20                        \\
                  \cline{3-5}
              \end{tabular}
          \end{center}

    \item \textbf{Identify the training F1 after a post-pruning of the given tree under a maximum depth of 1.}

          We start by drawing the decision tree after pruning.

          \begin{figure}[h!]
              \centering
              \includesvg[width=5cm]{assets/hw1-tree2.svg}
              \caption{Decision tree after pruning}
              \label{fig:ex2-tree-pruned}
          \end{figure}

          Then, we draw the training confusion matrix, according to the pruned tree.

          \begin{center}
              \begin{tabular}{|c|c|c|c|c|}
                  \cline{3-4}
                  \multicolumn{2}{c}{}                & \multicolumn{2}{|c|}{\textbf{Actual}} & \multicolumn{1}{c}{}                             \\
                  \cline{3-4}
                  \multicolumn{2}{c|}{}               & \textbf{Positive}                     & \textbf{Negative}    & \multicolumn{1}{c}{}      \\
                  \hline
                  \multirow{2}{*}{\textbf{Predicted}} & \textbf{Positive}                     & 5                    & 2                    & 7  \\
                  \cline{2-5}
                                                      & \textbf{Negative}                     & 6                    & 7                    & 13 \\
                  \hline
                  \multicolumn{2}{c|}{}               & 11                                    & 9                    & 20                        \\
                  \cline{3-5}
              \end{tabular}
          \end{center}

          The training F1 can be calculated by:

          \begin{equation}\label{ex2-f1}
              F_1 = \frac{1}{\frac{1}{2} \times \left(\frac{1}{P} + \frac{1}{R}\right)}
          \end{equation}

          Where \(P\) is precision and \(R\) is recall, which can be calculated by:

          \begin{equation}\label{ex2-precision-recall}
              \begin{array}{l r}
                  P = \frac{\text{true positives}}{\text{true positives} + \text{false positives}}
                   & R = \frac{\text{true positives}}{\text{true positives} + \text{false negatives}}
              \end{array}
          \end{equation}

          We have \textbf{5 true positives}, \textbf{6 false negatives} and \textbf{2 false positives}.\\
          As per \eqref{ex2-f1} and \eqref{ex2-precision-recall},

          \[
              \begin{aligned}
                  F_1 & = \frac{1}{\frac{1}{2} \times \left(\frac{5 + 2}{5} + \frac{5 + 6}{5}\right)} \\
                      & = \frac{5}{9} = 0.5556
              \end{aligned}
          \]

          Therefore, \(F_1 = \frac{5}{9} = 0.5556\).

    \item \textbf{Identify two different reasons as to why the left tree path was not further decomposed.}

          // TODO

    \item \textbf{Compute the information gain of variable \texttt{y1}.}

          The information gain of variable \texttt{y1} is given by

          \begin{equation}\label{ex4-ig}
              IG(y_1) = E(y_{out}) - E(y_{out} | y_1)
          \end{equation}

          The entropy of \(y_{out}\) is given by:

          \[
              E(y_{out}) = - \left(p(P) \log_2 \left(p(P)\right) + p(N) \log_2 \left(p(N)\right)\right)
          \]

          Since there are \textbf{11 positives} and \textbf{9 negatives}, we have the following proportions:

          \[
              \begin{array}{lcr}
                  p(P) = \frac{11}{20} & , & p(N) = \frac{9}{20}
              \end{array}
          \]

          Therefore, we can calculate $E(y_{out})$:

          \[
              \begin{aligned}
                  E(y_{out}) & = - \left(\frac{11}{20} \log_2\left(\frac{11}{20}\right) + \frac{9}{20} \log_2\left(\frac{9}{20}\right)\right) \\
                             & = 0.99277
              \end{aligned}
          \]

          The next step is calculating $E(y_{out} | y_1)$:

          \begin{equation}\label{ex4-e-yout-y1}
              E(y_{out} | y_1) = p(y_1 = A) E(y_{out} , y_1 = A) + p(y_1 = B) E(y_{out} , y_1 = B)
          \end{equation}

          Since there are \textbf{7 observations where \(y_1 = A\)}, \textbf{5 of which are positive} while \textbf{2 of which are negative},
          while remaining \textbf{13 observations have \(y_1 = B\)}, where \textbf{6 are positive} and \textbf{7 are negative},

          \[
              \begin{aligned}
                  p(y_1 = A)          & = \frac{7}{20}                                                                                               \\
                  p(y_1 = B)          & = \frac{13}{20}                                                                                              \\
                  E(y_{out}, y_1 = A) & = - \left(\frac{5}{7} \log_2\left(\frac{5}{7}\right) + \frac{2}{7} \log_2\left(\frac{2}{7}\right)\right)     \\
                                      & = 0.86312                                                                                                    \\
                  E(y_{out}, y_1 = B) & = - \left(\frac{6}{13} \log_2\left(\frac{6}{13}\right) + \frac{7}{13} \log_2\left(\frac{7}{13}\right)\right) \\
                                      & = 0.99573
              \end{aligned}
          \]

          Therefore, replacing these values on equation \eqref{ex4-e-yout-y1}, gives us:

          \[
              \begin{aligned}
                  E(y_{out} | y_1) & = \frac{7}{20} \times 0.86312 + \frac{13}{20} \times 0.99573 \\
                                   & = 0.94932
              \end{aligned}
          \]

          Finally, we can calculate the information gain, as per \eqref{ex4-ig},

          \[
              IG(y_{out}) = 0.99277 - 0.94932 = 0.04345
          \]

          Therefore, \(IG(y_{out}) = 0.04345\).
\end{enumerate}

\center\large{\textbf{Part II}: Programming}

\begin{enumerate}[leftmargin=\labelsep]
    \item {\bfseries Using sklearn, apply a stratified 70-30 training-testing split with a fixed seed
          \texttt{random\_state=1}, and assess in a single plot the training and testing accuracies of
          a decision tree with no depth limits (and remaining default behavior) for a varying number
          of selected features in \(\{5,10,40,100,250,700\}\).
          Feature selection should be performed before decision tree learning considering the
          discriminative power of the input variables according to mutual information criterion
          \texttt{(mutual\_info\_classif)}.
          }

          \begin{figure}[h!]
              \centering
              \includesvg[width=15cm]{assets/hw1-plot.svg}
              \caption{Accuracy of trained decision tree with no depth limits, applied to both a test and training sets, for a number of selected features.}
              \label{fig:ex1-plot}
          \end{figure}

    \item \textbf{Why training accuracy is persistently 1? Critically analyze the gathered results.}

          From the obtained results, we noticed that the training accuracy is always 1, regardless of the number of selected features.
          This is a result of how the decision trees learn.

          Since the question prompt tells us the tree does not have a depth limit, for each element in the training set, a new leaf will be created, which has its values as the path on the tree.
          Therefore, after the tree is trained, if we give the training set as the data set to test its accuracy, it'll know the correct path for all of the elements and knows how to classify them.
          This results in an accuracy of 1.

          However, if we test the model with a data set that it hasn't been trained on, we see its accuracy slightly decreases to around 0.8.
          This happens because it has never seen the values before, so it might have leaves that are not expanded enough to accurately classify the elements.
\end{enumerate}

\center\large{\textbf{Appendix}\vskip 0.3cm}

\lstinputlisting[language=Python]{assets/hw1-code.py}

\end{document}

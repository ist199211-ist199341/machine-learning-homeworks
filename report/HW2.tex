\documentclass[12pt]{article}
\usepackage[paper=letterpaper,margin=2cm]{geometry}
\usepackage{amsmath}
\usepackage{amssymb}
\usepackage{amsfonts}
\usepackage{newtxtext, newtxmath}
\usepackage{enumitem}
\usepackage{titling}
\usepackage[colorlinks=true]{hyperref}
\usepackage{multirow}
\usepackage{svg}
\usepackage{listings}
\usepackage{xcolor}
\usepackage{float}

\setlength{\droptitle}{-6em}

\definecolor{codegreen}{rgb}{0,0.6,0}
\definecolor{codegray}{rgb}{0.5,0.5,0.5}
\definecolor{codepurple}{rgb}{0.58,0,0.82}
\definecolor{backcolour}{rgb}{0.95,0.95,0.92}

\lstdefinestyle{mystyle}{
    commentstyle=\color{codegreen},
    keywordstyle=\color{magenta},
    numberstyle=\tiny\color{codegray},
    stringstyle=\color{codepurple},
    basicstyle=\ttfamily\footnotesize,
    breakatwhitespace=false,
    breaklines=true,
    captionpos=b,
    keepspaces=true,
    numbers=left,
    numbersep=5pt,
    showspaces=false,
    showstringspaces=false,
    showtabs=false,
    tabsize=2
}

\lstset{style=mystyle}

\title{\large{Aprendizagem 2022}\vskip 0.2cm Homework II -- Group 020}
\date{}
\author{Diogo Correia (99211) \and Tomás Esteves (99341)}
\begin{document}
\maketitle
\center\large{\vskip -1.0cm\textbf{Part I}: Pen and paper}

\begin{enumerate}[leftmargin=\labelsep]

    Four positive observations, \(
    \left\{
    \begin{pmatrix}
        A \\
        0
    \end{pmatrix}
    ,
    \begin{pmatrix}
        B \\
        1
    \end{pmatrix}
    ,
    \begin{pmatrix}
        A \\
        1
    \end{pmatrix}
    ,
    \begin{pmatrix}
        A \\
        0
    \end{pmatrix}
    \right\}
    \), and four negative observations, \(
    \left\{
    \begin{pmatrix}
        B \\
        0
    \end{pmatrix}
    ,
    \begin{pmatrix}
        B \\
        0
    \end{pmatrix}
    ,
    \begin{pmatrix}
        A \\
        1
    \end{pmatrix}
    ,
    \begin{pmatrix}
        B \\
        1
    \end{pmatrix}
    \right\}
    \), were collected.
    Consider the problem of classifying observations as positive or negative.

    \item {\bfseries
          Compute the recall of a distance-weighted \textit{k}NN with a \(k=5\)
          and distance \(d(x_1,x_2)=\text{Hamming}(x_1,x_2)+\frac{1}{2}\) using
          leave-one-out evaluation schema (i.e., when classifying one observation,
          use all remaining ones).
          }\\
          \vspace{0.5em}

          We start by calculating the distance between each observation.

          \begin{center}
              \begin{tabular}{c|cccccccc}
                  \(d(x_1,x_2)\) & \(x_1\)      & \(x_2\)      & \(x_3\)      & \(x_4\)      & \(x_5\)      & \(x_6\)      & \(x_7\)      & \(x_8\)      \\
                  \hline
                  \(x_1\)        & -            & 2.5          & \textbf{1.5} & \textbf{0.5} & \textbf{1.5} & \textbf{1.5} & \textbf{1.5} & 2.5          \\
                  \(x_2\)        & 2.5          & -            & \textbf{1.5} & 2.5          & \textbf{1.5} & \textbf{1.5} & \textbf{1.5} & \textbf{0.5} \\
                  \(x_3\)        & \textbf{1.5} & \textbf{1.5} & -            & \textbf{1.5} & 2.5          & 2.5          & \textbf{0.5} & \textbf{1.5} \\
                  \(x_4\)        & \textbf{0.5} & 2.5          & \textbf{1.5} & -            & \textbf{1.5} & \textbf{1.5} & \textbf{1.5} & 2.5          \\
                  \(x_5\)        & \textbf{1.5} & \textbf{1.5} & 2.5          & \textbf{1.5} & -            & \textbf{0.5} & 2.5          & \textbf{1.5} \\
                  \(x_6\)        & \textbf{1.5} & \textbf{1.5} & 2.5          & \textbf{1.5} & \textbf{0.5} & -            & 2.5          & \textbf{1.5} \\
                  \(x_7\)        & \textbf{1.5} & \textbf{1.5} & \textbf{0.5} & \textbf{1.5} & 2.5          & 2.5          & -            & \textbf{1.5} \\
                  \(x_8\)        & 2.5          & \textbf{0.5} & \textbf{1.5} & 2.5          & \textbf{1.5} & \textbf{1.5} & \textbf{1.5} & -
              \end{tabular}
          \end{center}

          We can predict the outcome for each observation, using the \textit{weighted mode}.

          \[
              \begin{aligned}
                  \hat{z_1} & = \text{weighted\_mode} \left(\left(\frac{1}{1.5}+\frac{1}{0.5}\right)P,\left(\frac{1}{1.5}+\frac{1}{1.5}+\frac{1}{1.5}\right)N\right) = P \\
                  \hat{z_2} & = \text{weighted\_mode} \left(\left(\frac{1}{1.5}\right)P,\left(\frac{1}{1.5}+\frac{1}{1.5}+\frac{1}{1.5}+\frac{1}{0.5}\right)N\right) = N \\
                  \hat{z_3} & = \text{weighted\_mode} \left(\left(\frac{1}{1.5}+\frac{1}{1.5}+\frac{1}{1.5}\right)P,\left(\frac{1}{0.5}+\frac{1}{1.5}\right)N\right) = N \\
                  \hat{z_4} & = \text{weighted\_mode} \left(\left(\frac{1}{0.5}+\frac{1}{1.5}\right)P,\left(\frac{1}{1.5}+\frac{1}{1.5}+\frac{1}{1.5}\right)N\right) = P \\
                  \hat{z_5} & = \text{weighted\_mode} \left(\left(\frac{1}{1.5}+\frac{1}{1.5}+\frac{1}{1.5}\right)P,\left(\frac{1}{0.5}+\frac{1}{1.5}\right)N\right) = N \\
                  \hat{z_6} & = \text{weighted\_mode} \left(\left(\frac{1}{1.5}+\frac{1}{1.5}+\frac{1}{1.5}\right)P,\left(\frac{1}{0.5}+\frac{1}{1.5}\right)N\right) = N \\
                  \hat{z_7} & = \text{weighted\_mode} \left(\left(\frac{1}{1.5}+\frac{1}{1.5}+\frac{1}{0.5}+\frac{1}{1.5}\right)P,\left(\frac{1}{1.5}\right)N\right) = P \\
                  \hat{z_8} & = \text{weighted\_mode} \left(\left(\frac{1}{0.5}+\frac{1}{1.5}\right)P,\left(\frac{1}{1.5}+\frac{1}{1.5}+\frac{1}{1.5}\right)N\right) = P
              \end{aligned}
          \]

          \begin{center}
              \begin{tabular}{c|c|c}
                  Observation & \(z\) & \(\hat{z}\) \\
                  \hline
                  \(x_1\)     & P     & P           \\
                  \(x_2\)     & P     & N           \\
                  \(x_3\)     & P     & N           \\
                  \(x_4\)     & P     & P           \\
                  \(x_5\)     & N     & N           \\
                  \(x_6\)     & N     & N           \\
                  \(x_7\)     & N     & P           \\
                  \(x_8\)     & N     & P
              \end{tabular}
          \end{center}

          Then, we draw the training confusion matrix, according to the obtained results.

          \begin{center}
              \begin{tabular}{|c|c|c|c|c|}
                  \cline{3-4}
                  \multicolumn{2}{c}{}                & \multicolumn{2}{|c|}{\textbf{Actual}} & \multicolumn{1}{c}{}                            \\
                  \cline{3-4}
                  \multicolumn{2}{c|}{}               & \textbf{Positive}                     & \textbf{Negative}    & \multicolumn{1}{c}{}     \\
                  \hline
                  \multirow{2}{*}{\textbf{Predicted}} & \textbf{Positive}                     & 2                    & 2                    & 4 \\
                  \cline{2-5}
                                                      & \textbf{Negative}                     & 2                    & 2                    & 4 \\
                  \hline
                  \multicolumn{2}{c|}{}               & 4                                     & 4                    & 8                        \\
                  \cline{3-5}
              \end{tabular}
          \end{center}

          We can now calculate the recall:

          \[
              \text{recall} = \frac{\#\text{true positives}}{\#\text{positives}} = \frac{2}{4} = 0.5
          \]

\end{enumerate}

\center\large{\textbf{Part II}: Programming}

\begin{enumerate}[leftmargin=\labelsep]
    \item TODO
\end{enumerate}

\center\large{\textbf{Appendix}\vskip 0.3cm}

\end{document}

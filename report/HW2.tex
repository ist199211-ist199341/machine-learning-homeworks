\documentclass[12pt]{article}
\usepackage[paper=letterpaper,margin=2cm]{geometry}
\usepackage{amsmath}
\usepackage{amssymb}
\usepackage{amsfonts}
\usepackage{newtxtext, newtxmath}
\usepackage{enumitem}
\usepackage{titling}
\usepackage[colorlinks=true]{hyperref}
\usepackage{multirow}
\usepackage{svg}
\usepackage{listings}
\usepackage{xcolor}
\usepackage{float}
\usepackage{graphicx}
\usepackage{subcaption}

\setlength{\droptitle}{-6em}

\definecolor{codegreen}{rgb}{0,0.6,0}
\definecolor{codegray}{rgb}{0.5,0.5,0.5}
\definecolor{codepurple}{rgb}{0.58,0,0.82}
\definecolor{backcolour}{rgb}{0.95,0.95,0.92}

\lstdefinestyle{mystyle}{
    commentstyle=\color{codegreen},
    keywordstyle=\color{magenta},
    numberstyle=\tiny\color{codegray},
    stringstyle=\color{codepurple},
    basicstyle=\ttfamily\footnotesize,
    breakatwhitespace=false,
    breaklines=true,
    captionpos=b,
    keepspaces=true,
    numbers=left,
    numbersep=5pt,
    showspaces=false,
    showstringspaces=false,
    showtabs=false,
    tabsize=2
}

\lstset{style=mystyle}

\title{\large{Aprendizagem 2022}\vskip 0.2cm Homework II -- Group 020}
\date{}
\author{Diogo Correia (99211) \and Tomás Esteves (99341)}
\begin{document}
\maketitle
\begin{center}
    \large{\vskip -1.0cm\textbf{Part I}: Pen and paper}
\end{center}

Four positive observations, \(
\left\{
\begin{pmatrix}
    A \\
    0
\end{pmatrix}
,
\begin{pmatrix}
    B \\
    1
\end{pmatrix}
,
\begin{pmatrix}
    A \\
    1
\end{pmatrix}
,
\begin{pmatrix}
    A \\
    0
\end{pmatrix}
\right\}
\), and four negative observations, \(
\left\{
\begin{pmatrix}
    B \\
    0
\end{pmatrix}
,
\begin{pmatrix}
    B \\
    0
\end{pmatrix}
,
\begin{pmatrix}
    A \\
    1
\end{pmatrix}
,
\begin{pmatrix}
    B \\
    1
\end{pmatrix}
\right\}
\), were collected.
Consider the problem of classifying observations as positive or negative.

\begin{enumerate}[leftmargin=\labelsep]
    \item {\bfseries
          Compute the recall of a distance-weighted \textit{k}NN with a \(k=5\)
          and distance \(d(x_1,x_2)=\text{Hamming}(x_1,x_2)+\frac{1}{2}\) using
          leave-one-out evaluation schema (i.e., when classifying one observation,
          use all remaining ones).
          }\\
          \vspace{0.5em}

          We start by calculating the distance between each observation.

          \begin{center}
              \begin{tabular}{c|cccccccc}
                  \(d(x_i,x_j)\) & \(x_1\)      & \(x_2\)      & \(x_3\)      & \(x_4\)      & \(x_5\)      & \(x_6\)      & \(x_7\)      & \(x_8\)      \\
                  \hline
                  \(x_1\)        & -            & 2.5          & \textbf{1.5} & \textbf{0.5} & \textbf{1.5} & \textbf{1.5} & \textbf{1.5} & 2.5          \\
                  \(x_2\)        & 2.5          & -            & \textbf{1.5} & 2.5          & \textbf{1.5} & \textbf{1.5} & \textbf{1.5} & \textbf{0.5} \\
                  \(x_3\)        & \textbf{1.5} & \textbf{1.5} & -            & \textbf{1.5} & 2.5          & 2.5          & \textbf{0.5} & \textbf{1.5} \\
                  \(x_4\)        & \textbf{0.5} & 2.5          & \textbf{1.5} & -            & \textbf{1.5} & \textbf{1.5} & \textbf{1.5} & 2.5          \\
                  \(x_5\)        & \textbf{1.5} & \textbf{1.5} & 2.5          & \textbf{1.5} & -            & \textbf{0.5} & 2.5          & \textbf{1.5} \\
                  \(x_6\)        & \textbf{1.5} & \textbf{1.5} & 2.5          & \textbf{1.5} & \textbf{0.5} & -            & 2.5          & \textbf{1.5} \\
                  \(x_7\)        & \textbf{1.5} & \textbf{1.5} & \textbf{0.5} & \textbf{1.5} & 2.5          & 2.5          & -            & \textbf{1.5} \\
                  \(x_8\)        & 2.5          & \textbf{0.5} & \textbf{1.5} & 2.5          & \textbf{1.5} & \textbf{1.5} & \textbf{1.5} & -
              \end{tabular}
          \end{center}

          We can predict the outcome for each observation, using the \textit{weighted mode}.

          \[
              \begin{aligned}
                  \hat{z_1} & = \text{weighted\_mode} \left(\left(\frac{1}{1.5}+\frac{1}{0.5}\right)P,\left(\frac{1}{1.5}+\frac{1}{1.5}+\frac{1}{1.5}\right)N\right) = P \\
                  \hat{z_2} & = \text{weighted\_mode} \left(\left(\frac{1}{1.5}\right)P,\left(\frac{1}{1.5}+\frac{1}{1.5}+\frac{1}{1.5}+\frac{1}{0.5}\right)N\right) = N \\
                  \hat{z_3} & = \text{weighted\_mode} \left(\left(\frac{1}{1.5}+\frac{1}{1.5}+\frac{1}{1.5}\right)P,\left(\frac{1}{0.5}+\frac{1}{1.5}\right)N\right) = N \\
                  \hat{z_4} & = \text{weighted\_mode} \left(\left(\frac{1}{0.5}+\frac{1}{1.5}\right)P,\left(\frac{1}{1.5}+\frac{1}{1.5}+\frac{1}{1.5}\right)N\right) = P \\
                  \hat{z_5} & = \text{weighted\_mode} \left(\left(\frac{1}{1.5}+\frac{1}{1.5}+\frac{1}{1.5}\right)P,\left(\frac{1}{0.5}+\frac{1}{1.5}\right)N\right) = N \\
                  \hat{z_6} & = \text{weighted\_mode} \left(\left(\frac{1}{1.5}+\frac{1}{1.5}+\frac{1}{1.5}\right)P,\left(\frac{1}{0.5}+\frac{1}{1.5}\right)N\right) = N \\
                  \hat{z_7} & = \text{weighted\_mode} \left(\left(\frac{1}{1.5}+\frac{1}{1.5}+\frac{1}{0.5}+\frac{1}{1.5}\right)P,\left(\frac{1}{1.5}\right)N\right) = P \\
                  \hat{z_8} & = \text{weighted\_mode} \left(\left(\frac{1}{0.5}+\frac{1}{1.5}\right)P,\left(\frac{1}{1.5}+\frac{1}{1.5}+\frac{1}{1.5}\right)N\right) = P
              \end{aligned}
          \]

          \begin{center}
              \begin{tabular}{c|c|c}
                  Observation & \(z\) & \(\hat{z}\) \\
                  \hline
                  \(x_1\)     & P     & P           \\
                  \(x_2\)     & P     & N           \\
                  \(x_3\)     & P     & N           \\
                  \(x_4\)     & P     & P           \\
                  \(x_5\)     & N     & N           \\
                  \(x_6\)     & N     & N           \\
                  \(x_7\)     & N     & P           \\
                  \(x_8\)     & N     & P
              \end{tabular}
          \end{center}

          Then, we draw the training confusion matrix, according to the obtained results.

          \begin{center}
              \begin{tabular}{|c|c|c|c|c|}
                  \cline{3-4}
                  \multicolumn{2}{c}{}                & \multicolumn{2}{|c|}{\textbf{Actual}} & \multicolumn{1}{c}{}                            \\
                  \cline{3-4}
                  \multicolumn{2}{c|}{}               & \textbf{Positive}                     & \textbf{Negative}    & \multicolumn{1}{c}{}     \\
                  \hline
                  \multirow{2}{*}{\textbf{Predicted}} & \textbf{Positive}                     & 2                    & 2                    & 4 \\
                  \cline{2-5}
                                                      & \textbf{Negative}                     & 2                    & 2                    & 4 \\
                  \hline
                  \multicolumn{2}{c|}{}               & 4                                     & 4                    & 8                        \\
                  \cline{3-5}
              \end{tabular}
          \end{center}

          We can now calculate the recall:

          \[
              \text{recall} = \frac{\#\text{true positives}}{\#\text{positives}} = \frac{2}{4} = 0.5
          \]

\end{enumerate}

{\bfseries
An additional positive observation was acquired, \(
\begin{pmatrix}
    B \\
    0
\end{pmatrix}
\), and a third variable \(y_3\) was independently monitored, yielding estimates
\(
y_3 | P = \left\{1.2, 0.8, 0.5, 0.9, 0.8\right\}
\) and \(
y_3 | N = \left\{1, 0.9, 1.2, 0.8\right\}
\).
}

\begin{enumerate}[leftmargin=\labelsep,resume]
    \item {\bfseries Considering the nine training observations, learn a Bayesian classifier assuming:
          \begin{enumerate}[label=(\roman*)]
              \item \(y_1\) and \(y_2\) are dependent;
              \item \(\left\{y_1, y_2\right\}\) and \(\left\{y_3\right\}\) variable sets
                    are independent and equally important;
              \item \(y_3\) is normally distributed.
          \end{enumerate}
          Show all parameters.
          }

          To better organize the information we have, we'll start by creating a table
          with all the training observations.

          \begin{center}
              \begin{tabular}{c|cccc}
                  Observation & \(y_1\) & \(y_2\) & \(y_3\) & class \\
                  \hline
                  \(x_1\)     & A       & 0       & 1.2     & P     \\
                  \(x_2\)     & B       & 1       & 0.8     & P     \\
                  \(x_3\)     & A       & 1       & 0.5     & P     \\
                  \(x_4\)     & A       & 0       & 0.9     & P     \\
                  \(x_5\)     & B       & 0       & 0.8     & P     \\
                  \(x_6\)     & B       & 0       & 1.0     & N     \\
                  \(x_7\)     & B       & 0       & 0.9     & N     \\
                  \(x_8\)     & A       & 1       & 1.2     & N     \\
                  \(x_9\)     & B       & 1       & 0.8     & N
              \end{tabular}
          \end{center}

          As stated by the question prompt, variable sets \(\left\{y_1, y_2\right\}\)
          and \(\left\{y_3\right\}\) are independent and equally important.
          Since we have two independent sets, we'll train a Naïve Bayes classifier.

          We'll refer to the outcome, which can be positive (pos) or
          negative (neg), as class.

          To estimate $p(\text{class} | y_1, y_2, y_3)$, we can use Bayes' theorem:

          \begin{equation}\label{ex2-bayes1}
              p(\text{class}| y_1, y_2, y_3) = \frac{p(y_1, y_2, y_3 | \text{class}) \times p(\text{class})}{p(y_1, y_2, y_3)}
          \end{equation}

          Since we know $\left\{y_1, y_2\right\}$ and $\left\{y_3\right\}$ are independent,
          we can rewrite $p(y_1, y_2, y_3)$ as $p(y_1, y_2) \cdot p(y_3)$.
          Rewriting \eqref{ex2-bayes1} with this, results in:

          \begin{equation}\label{ex2-bayes2}
              p(\text{class}| y_1, y_2, y_3) = \frac{p(y_1, y_2 | \text{class}) p(y_3 | \text{class}) \times p(\text{class})}{p(y_1, y_2)p(y_3)}
          \end{equation}

          Given a new observation $D$, we are able to classify it by calculating
          $p(\text{class}|D)$ for all classes and selecting the class with the
          highest probability as our prediction.

          $$
              \begin{aligned}
                  \hat{z} & = \underset{c \in \{pos, neg\}}{\text{arg max}} \medspace p(\text{c} | D)                                                                                                                           \\
                          & = \underset{c \in \{pos, neg\}}{\text{arg max}} \medspace \frac{p(y_1, y_2 | c) p(y_3 | c) \times p(c)}{p(y_1, y_2)p(y_3)}                                                                          \\
                          & = \underset{c \in \{pos, neg\}}{\text{arg max}} \medspace p(y_1, y_2 | c) p(y_3 | c) p(c)                                  & \parbox{15em}{since we can remove parameters that don't depend on $c$}
              \end{aligned}
          $$


          We can therefore start calculating all these parameters.

          \textbf{Note:} Even though $p(y_1, y_2)$ and $p(y_3)$ are not necessary
          to apply the model, we'll still calculate them for the sake of showing
          all parameters.

          Calculating $p(\text{pos})$, $p(\text{neg})$ and all parameters involving $y_1$ and
          $y_2$ is straightforward, since they can be infered from the table.

          There are 5 positive and 4 negative observations, out of a total of 9.
          Therefore,

          \[
              \begin{array}{cc}
                  p(\text{pos}) = \frac{5}{9} &
                  p(\text{neg}) = \frac{4}{9}
              \end{array}
          \]

          For the four possible combinations of $y_1$ and $y_2$, and following
          the same logic as above,

          \[
              \begin{array}{cc}
                  p(y_1 = A, \ y_2 = 0) = \frac{2}{9}, &
                  p(y_1 = A, \ y_2 = 1) = \frac{2}{9}    \\[\medskipamount]
                  p(y_1 = B, \ y_2 = 0) = \frac{3}{9}, &
                  p(y_1 = B, \ y_2 = 1) = \frac{2}{9}
              \end{array}
          \]

          Finally, considering each class and the four possible combinations
          of $y_1$ and $y_2$, we can use the table to calculate the following:

          \[
              \begin{array}{cc}
                  p(y_1 = A, \ y_2 = 0 \ | \text{pos}) = \frac{2}{5}, &
                  p(y_1 = A, \ y_2 = 1 \ | \text{pos}) = \frac{1}{5}    \\[\medskipamount]
                  p(y_1 = B, \ y_2 = 0 \ | \text{pos}) = \frac{1}{5}, &
                  p(y_1 = B, \ y_2 = 1 \ | \text{pos}) = \frac{1}{5}
              \end{array}
          \]

          \[
              \begin{array}{cc}
                  p(y_1 = A, \ y_2 = 0 \ | \text{neg}) = \frac{0}{4}, &
                  p(y_1 = A, \ y_2 = 1 \ | \text{neg}) = \frac{1}{4}    \\[\medskipamount]
                  p(y_1 = B, \ y_2 = 0 \ | \text{neg}) = \frac{2}{4}, &
                  p(y_1 = B, \ y_2 = 1 \ | \text{neg}) = \frac{1}{4}
              \end{array}
          \]

          Calculating now the parameters related to the variable set $\left\{y_3\right\}$. We know that $y_3$ follows a Normal Distribution.
          Therefore,

          \begin{equation}\label{ex2-normal}
              p\left(y_3 | \mu, \sigma^2\right)
              = \mathcal{N}\left(y_3 | \mu, \sigma^2\right)
              = \frac{1}{\sqrt{2 \pi} \sigma} e ^{- \frac{\left(y_3- \mu\right)^2}{2 \sigma^2}}
          \end{equation}

          We can use the observations we have to approximate a value for the
          mean ($\mu$) and variance ($\sigma^2$).

          $$
              \begin{aligned}
                  \mu & = \frac{\sum^{9}_{i=1} y_{3,i}}{9}                              \\
                      & = \frac{1.2 + 0.8 + 0.5 + 0.9 + 0.8 + 1.0 + 0.9 + 1.2 + 0.8}{9} \\
                      & = 0.9
              \end{aligned}
          $$

          $$
              \begin{aligned}
                  \sigma^2 & = \frac{1}{9 - 1} \sum^9_{i=1} \left(y_{3,i} - \mu\right)^2 \\
                           & = \dots                                                     \\
                           & = 0.0475
              \end{aligned}
          $$

          Therefore, $P(y_3) \sim \mathcal{N}(y_3 | \mu = 0.9, \sigma^2 = 0.0475)$.

          We can repeat the process for both classes (positive and negative).
          Starting with positive:

          $$
              \begin{array}{c|c}
                  \begin{aligned}
                      \mu & = \frac{\sum^{5}_{i=1} y_{3,i}}{5}      \\
                          & = \frac{1.2 + 0.8 + 0.5 + 0.9 + 0.8}{5} \\
                          & = 0.84
                  \end{aligned}
                  \quad &
                  \quad
                  \begin{aligned}
                      \sigma^2 & = \frac{1}{5 - 1} \sum^5_{i=1} \left(y_{3,i} - \mu\right)^2 \\
                               & = \dots                                                     \\
                               & = 0.063
                  \end{aligned}
              \end{array}
          $$

          Therefore, $P(y_3|\text{pos}) \sim \mathcal{N}(y_3 | \mu = 0.84, \sigma^2 = 0.063)$.

          And with negative:

          $$
              \begin{array}{c|c}
                  \begin{aligned}
                      \mu & = \frac{\sum^{4}_{i=1} y_{3,i}}{4} \\
                          & = \frac{1.0 + 0.9 + 1.2 + 0.8}{4}  \\
                          & = 0.975
                  \end{aligned}
                  \quad &
                  \quad
                  \begin{aligned}
                      \sigma^2 & = \frac{1}{4 - 1} \sum^4_{i=1} \left(y_{3,i} - \mu\right)^2 \\
                               & = \dots                                                     \\
                               & = 0.0292
                  \end{aligned}
              \end{array}
          $$

          Therefore, $P(y_3|\text{neg}) \sim \mathcal{N}(y_3 | \mu = 0.975, \sigma^2 = 0.0292)$.

          We now have all parameters necessary to apply the Naïve Bayes classifier.

\end{enumerate}

\vspace{3em}

{\bfseries
    Considering three testing observations,
    \(
    \left\{
    \left(
    \begin{pmatrix}
        A \\
        1 \\
        0.8
    \end{pmatrix},
    \text{Positive}
    \right),
    \left(
    \begin{pmatrix}
        B \\
        1 \\
        1
    \end{pmatrix},
    \text{Positive}
    \right),
    \left(
    \begin{pmatrix}
        B \\
        0 \\
        0.9
    \end{pmatrix},
    \text{Negative}
    \right),
    \right\}
    \)
}

\begin{enumerate}[leftmargin=\labelsep,resume]
    \item \textbf{Under a MAP assumption, compute $P(\text{Positive} | x)$ of each testing observation.}

          In the previous question, we determined the expression of $p(\text{class} | y_1,y_2,y_3)$ \eqref{ex2-bayes2}.
          We can now use it and replace the values for each observation.

          \begin{equation}\label{ex3-bayes}
              p(\text{pos}| y_1, y_2, y_3) = \frac{p(y_1, y_2 | \text{pos}) p(y_3 | \text{pos}) \times p(\text{pos})}{p(y_1, y_2)p(y_3)}
          \end{equation}

          However, since $p(\text{pos}|y_1,y_2,y_3) + p(\text{neg}|y_1,y_2,y_3)$ must be 1, we need to normalize the values.

          Therefore, from \eqref{ex3-bayes},

          \begin{equation}\label{ex3-bayes-normalized}
              \begin{aligned}
                  p(\text{pos}| y_1, y_2, y_3) & = \frac{
                      \frac{p(y_1, y_2 | \text{pos}) p(y_3 | \text{pos}) \times p(\text{pos})}{p(y_1, y_2)p(y_3)}
                  }{
                      \frac{p(y_1, y_2 | \text{pos}) p(y_3 | \text{pos}) \times p(\text{pos})}{p(y_1, y_2)p(y_3)} +
                      \frac{p(y_1, y_2 | \text{neg}) p(y_3 | \text{neg}) \times p(\text{neg})}{p(y_1, y_2)p(y_3)}
                  }                                       \\
                                               & = \frac{
                      p(y_1, y_2 | \text{pos}) p(y_3 | \text{pos}) \times p(\text{pos})
                  }{
                      p(y_1, y_2 | \text{pos}) p(y_3 | \text{pos}) \times p(\text{pos}) +
                      p(y_1, y_2 | \text{neg}) p(y_3 | \text{neg}) \times p(\text{neg})
                  }
              \end{aligned}
          \end{equation}

          Replacing the values on \eqref{ex3-bayes-normalized} for each observation,
          using \eqref{ex2-normal} to calculate the values of $p(y_3)$,
          $p(y_3|\text{pos})$ and $p(y_3|\text{neg})$,

          $$
              \begin{aligned}
                   & p(\text{pos}, y_1=A, y_2=1, y_3=0.8)                                                     \\
                   & = \frac{p(y_1=A, y_2=1 | \text{pos}) p(y_3=0.8 | \text{pos}) \times p(\text{pos})}{
                      p(y_1=A, y_2=1 | \text{pos}) p(y_3=0.8 | \text{pos}) \times p(\text{pos}) +
                      p(y_1=A, y_2=1 | \text{neg}) p(y_3=0.8 | \text{neg}) \times p(\text{neg})
                  }                                                                                           \\
                   & \approx \frac
                  {\frac{1}{5} \times 1.569 \times \frac{5}{9}}
                  {\frac{1}{5} \times 1.569 \times \frac{5}{9} + \frac{1}{4} \times 1.382 \times \frac{4}{9}} \\
                   & \approx 0.531
              \end{aligned}
          $$

          $$
              \begin{aligned}
                   & p(\text{pos}, y_1=B, y_2=1, y_3=1.0)                                                     \\
                   & = \frac{p(y_1=B, y_2=1 | \text{pos}) p(y_3=1.0 | \text{pos}) \times p(\text{pos})}{
                      p(y_1=B, y_2=1 | \text{pos}) p(y_3=1.0 | \text{pos}) \times p(\text{pos}) +
                      p(y_1=B, y_2=1 | \text{neg}) p(y_3=1.0 | \text{neg}) \times p(\text{neg})
                  }                                                                                           \\
                   & \approx \frac
                  {\frac{1}{5} \times 1.297 \times \frac{5}{9}}
                  {\frac{1}{5} \times 1.297 \times \frac{5}{9} + \frac{1}{4} \times 2.311 \times \frac{4}{9}} \\
                   & \approx 0.360
              \end{aligned}
          $$

          $$
              \begin{aligned}
                   & p(\text{pos}, y_1=B, y_2=0, y_3=0.9)                                                     \\
                   & = \frac{p(y_1=B, y_2=0 | \text{pos}) p(y_3=0.9 | \text{pos}) \times p(\text{pos})}{
                      p(y_1=B, y_2=0 | \text{pos}) p(y_3=0.9 | \text{pos}) \times p(\text{pos}) +
                      p(y_1=B, y_2=0 | \text{neg}) p(y_3=0.9 | \text{neg}) \times p(\text{neg})
                  }                                                                                           \\
                   & \approx \frac
                  {\frac{1}{5} \times 1.545 \times \frac{5}{9}}
                  {\frac{1}{5} \times 1.545 \times \frac{5}{9} + \frac{2}{4} \times 2.121 \times \frac{4}{9}} \\
                   & \approx 0.267
              \end{aligned}
          $$

    \item {\bfseries
          Given a binary class variable, the default decision threshold of $\theta = 0.5$,
          $$
              f(x|\theta) = \begin{cases}
                  \text{Positive} & P(\text{Positive}|x) > \theta \\
                  \text{Negative} & \text{otherwise}
              \end{cases}
          $$
          can be adjusted.
          Which decision threshold - 0.3, 0.5 or 0.7 - optimizes testing accuracy?
          }

          Considering that, of the three testing observations,
          the two positive ones had a $P(\text{Positive}|x)$ value
          of 0.531 and 0.360, which are both greater than 0.3
          and the only negative observation is 0.267, which is lesser
          than 0.3 (and 0.5 and 0.7).

          Therefore, of the three given thresholds (0.3, 0.5 and 0.7),
          the only one that correctly classifies all three observations
          is \textbf{0.3}. This results in an accuracy of 100\%.


\end{enumerate}

\begin{center}
    \large{\textbf{Part II}: Programming and critical analysis}
\end{center}

Considering the \texttt{pd\_speech.arff} dataset available at the course website.

\begin{enumerate}[leftmargin=\labelsep,resume]
    \item {\bfseries
          Using \texttt{sklearn}, considering a 10-fold stratified cross validation
          (\texttt{random=3}), plot que cumulative testing confusion matrices of
          \textit{k}NN (uniform weights, $k = 5$, Euclidean distance) and Naïve Bayes
          (Gaussian assumption). Use all remaining classifier parameters as default.
          }

          \begin{figure}[H]
              \centering
              \begin{subfigure}{0.49\textwidth}
                  \centering
                  \includesvg[width = \textwidth]{assets/hw2-knn-confusion-matrix.svg}
                  \caption{\textit{k}NN Confusion Matrix}
                  \label{fig:confusion-knn}
              \end{subfigure}
              \begin{subfigure}{0.49\textwidth}
                  \centering
                  \includesvg[width = \textwidth]{assets/hw2-naive-bayes-confusion-matrix.svg}
                  \caption{Naïve Bayes Confusion Matrix}
                  \label{fig:confusion-naive-bayes}
              \end{subfigure}
              \caption{Confusion Matrices for \textit{k}NN and Naïve Bayes}
              \label{fig:confusion-matrices}
          \end{figure}
\end{enumerate}

\center\large{\textbf{Appendix}\vskip 0.3cm}

\end{document}

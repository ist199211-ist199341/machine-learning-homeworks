\documentclass[12pt]{article}
\usepackage[paper=letterpaper,margin=2cm]{geometry}
\usepackage{amsmath}
\usepackage{amssymb}
\usepackage{amsfonts}
\usepackage{newtxtext, newtxmath}
\usepackage{enumitem}
\usepackage{titling}
\usepackage[colorlinks=true]{hyperref}
\usepackage{multirow}
\usepackage{svg}
\usepackage{listings}
\usepackage{xcolor}
\usepackage{float}
\usepackage{graphicx}
\usepackage{subcaption}
\usepackage{paracol}

\setlength{\droptitle}{-6em}

\definecolor{codegreen}{rgb}{0,0.6,0}
\definecolor{codegray}{rgb}{0.5,0.5,0.5}
\definecolor{codepurple}{rgb}{0.58,0,0.82}
\definecolor{backcolour}{rgb}{0.95,0.95,0.92}

% TEXT COLORS
% https://coolors.co/4c6085-1588e0-0cac8c-f25f5c-b89300
\definecolor{cgray}{HTML}{4c6085}
\definecolor{cblue}{HTML}{1588e0}
\definecolor{cgreen}{HTML}{0cac8c}
\definecolor{cred}{HTML}{f25f5c}
\definecolor{cyellow}{HTML}{b89300}

% BACKGROUND COLORS
% https://coolors.co/8a9cbc-7cc0f3-66f4d8-f6908e-ffe270
\definecolor{bgray}{HTML}{8a9cbc}
\definecolor{bblue}{HTML}{7cc0f3}
\definecolor{bgreen}{HTML}{66f4d8}
\definecolor{bred}{HTML}{f6908e}
\definecolor{byellow}{HTML}{ffe270}

\definecolor{linkcolor}{HTML}{f57429}
\definecolor{questioncolor}{HTML}{444444}
\definecolor{listingbackground}{HTML}{f5f5f5}

\hypersetup{
    allcolors=linkcolor
}

\setlength{\columnseprule}{1pt}
\def\columnseprulecolor{\color{questioncolor}}

\lstdefinestyle{mystyle}{
    commentstyle=\color{codegreen},
    keywordstyle=\color{magenta},
    numberstyle=\tiny\color{codegray},
    stringstyle=\color{codepurple},
    basicstyle=\ttfamily\footnotesize,
    backgroundcolor=\color{listingbackground},
    breakatwhitespace=false,
    breaklines=true,
    captionpos=b,
    keepspaces=true,
    numbers=left,
    numbersep=5pt,
    showspaces=false,
    showstringspaces=false,
    showtabs=false,
    tabsize=2
}

\lstset{style=mystyle}

\title{\large{Aprendizagem 2022}\vskip 0.2cm Homework III -- Group 020}
\date{}
\author{Diogo Correia (99211) \and Tomás Esteves (99341)}
\begin{document}
\maketitle
\begin{center}
    \large{\vskip -1.0cm\textbf{Part I}: Pen and paper}
\end{center}

{
\color{questioncolor}\bfseries
\noindent
Consider the problem of learning a regression model from 5 univariate observations\\
$\left((0.8), (1), (1.2), (1.4), (1.6)\right)$ with targets $\left(24, 20, 10, 13, 12\right)$.
}

\begin{enumerate}[leftmargin=\labelsep]
    \item {\color{questioncolor}\bfseries
          Consider the basis function, $\phi_j (x) = x^j$, for performing a 3-order
          polynomial regression,

          $$
              \hat{z}(x, w) = \sum_{j=0}^{3} w_j \phi_j (x)
              = w_0 + w_1 x + w_2 x^2 + w_3 x^3
          $$

          Learn the Ridge regression ($l_2$ regularization) on the transformed data
          space using the closed form solution with $\lambda = 2$.

          \textit{Hint:} use \texttt{numpy} matrix operations (e.g., \texttt{linalg.pinv} for inverse)
          to validate your calculus.
          }\\
          \vspace{0.5em}

          Considering the 5 observations, we can calculate the value of
          $\phi_0 (x)$, $\phi_1 (x)$, $\phi_2 (x)$ and $\phi_3 (x)$ for each
          of them.

          \begin{center}
              \captionsetup{type=table}
              \begin{tabular}{c|cccc}
                  $x$ & $\phi_0(x)$ & $\phi_1(x)$ & $\phi_2(x)$ & $\phi_3(x)$ \\
                  \hline
                  0.8 & 1           & 0.8         & 0.64        & 0.512       \\
                  1   & 1           & 1           & 1           & 1           \\
                  1.2 & 1           & 1.2         & 1.44        & 1.728       \\
                  1.4 & 1           & 1.4         & 1.96        & 2.744       \\
                  1.6 & 1           & 1.6         & 2.56        & 4.096
              \end{tabular}
              \captionof{table}{Value of $\phi_j(x)$ for each observation, $j=0\dots3$}
              \label{ex1-phi-table}
          \end{center}

          We shall consider $z$ as the vector of targets for the respective observations, $
              z = \begin{bmatrix}
                  24 & 20 & 10 & 13 & 12
              \end{bmatrix}^T
          $.

          Our goal is to perform a regression where the prediction is given by

          \begin{equation}\label{ex1-z-hat}
              \hat{z}(x, w)
              = \textcolor{cblue}{w_0} +
              \textcolor{cred}{w_1}x +
              \textcolor{cgreen}{w_2}x^2 +
              \textcolor{cyellow}{w_3}x^3
          \end{equation}

          We know the parameters in the Ridge regression are given by \eqref{ex1-ridge},
          where $X$ is the matrix after applying $\phi_j(x)$ to the observations,
          as per Table \ref{ex1-phi-table}.

          \begin{equation}\label{ex1-ridge}
              w = \left(X^T X + \lambda I\right)^{-1} \cdot X^T z
          \end{equation}

          $$
              X = \begin{bmatrix}{}
                  1.0000 & 0.8000 & 0.6400 & 0.5120 \\
                  1.0000 & 1.0000 & 1.0000 & 1.0000 \\
                  1.0000 & 1.2000 & 1.4400 & 1.7280 \\
                  1.0000 & 1.4000 & 1.9600 & 2.7440 \\
                  1.0000 & 1.6000 & 2.5600 & 4.0960
              \end{bmatrix}
              \quad
              \quad
              X^T =\begin{bmatrix}{}
                  1.0000 & 1.0000 & 1.0000 & 1.0000 & 1.0000 \\
                  0.8000 & 1.0000 & 1.2000 & 1.4000 & 1.6000 \\
                  0.6400 & 1.0000 & 1.4400 & 1.9600 & 2.5600 \\
                  0.5120 & 1.0000 & 1.7280 & 2.7440 & 4.0960
              \end{bmatrix}
          $$

          We can then calculate the remaining values, until we get the vector $w$.

          $$
              \begin{aligned}
                  X^T X                                    & = \begin{bmatrix}
                                                                   5.0000  & 6.0000  & 7.6000  & 10.0800 \\
                                                                   6.0000  & 7.6000  & 10.0800 & 13.8784 \\
                                                                   7.6000  & 10.0800 & 13.8784 & 19.6800 \\
                                                                   10.0800 & 13.8784 & 19.6800 & 28.5549
                                                               \end{bmatrix}                                                                                   \\
                  X^T X + \textcolor{cblue}{2} I           & = \begin{bmatrix}{}
                                                                   \colorbox{bblue}{$ 7.0000$} & 6.0000                      & 7.6000                       & 10.0800                      \\
                                                                   6.0000                      & \colorbox{bblue}{$ 9.6000$} & 10.0800                      & 13.8784                      \\
                                                                   7.6000                      & 10.0800                     & \colorbox{bblue}{$ 15.8784$} & 19.6800                      \\
                                                                   10.0800                     & 13.8784                     & 19.6800                      & \colorbox{bblue}{$ 30.5549$}
                                                               \end{bmatrix} \\
                  \left(X^T X + 2I\right) ^ {-1}           & = \begin{bmatrix}{}
                                                                   0.3417  & -0.1214 & -0.0749 & -0.0093 \\
                                                                   -0.1214 & 0.3892  & -0.0967 & -0.0745 \\
                                                                   -0.0749 & -0.0967 & 0.3726  & -0.1714 \\
                                                                   -0.0093 & -0.0745 & -0.1714 & 0.1800
                                                               \end{bmatrix}                                                                                   \\
                  \left(X^T X + 2I\right) ^ {-1} X^T       & = \begin{bmatrix}{}
                                                                   0.1918  & 0.1360  & 0.0720  & -0.0007 & -0.0825 \\
                                                                   0.0899  & 0.0966  & 0.0777  & 0.0297  & -0.0512 \\
                                                                   -0.0015 & 0.0296  & 0.0495  & 0.0498  & 0.0224  \\
                                                                   -0.0864 & -0.0751 & -0.0344 & 0.0445  & 0.1701
                                                               \end{bmatrix}                                                                         \\
                  w = \left(X^T X + 2I\right) ^ {-1} X^T z & = \begin{bmatrix}{}
                                                                   \textcolor{cblue}{7.0451} & \textcolor{cred}{4.6409} & \textcolor{cgreen}{1.9673} & \textcolor{cyellow}{-1.3009}
                                                               \end{bmatrix}^T
              \end{aligned}
          $$

          Having calculated the vector $w$, we now know the values of each
          weight in the regression.

          $$
              \begin{array}{cccc}
                  \textcolor{cblue}{w_0 = 7.0451}  &
                  \textcolor{cred}{w_1 = 4.6409}   &
                  \textcolor{cgreen}{w_2 = 1.9673} &
                  \textcolor{cyellow}{w_3 = -1.3009}
              \end{array}
          $$

          Replacing them in \eqref{ex1-z-hat} gives us the regression expression,

          \begin{equation}\label{ex1-final-z-hat}
              \hat{z}(x)
              = \textcolor{cblue}{7.0451} +
              \textcolor{cred}{4.6409}x +
              \textcolor{cgreen}{1.9673}x^2
              \textcolor{cyellow}{-1.3009}x^3
          \end{equation}

    \item {\color{questioncolor}\bfseries
          Compute the training RMSE for the learnt regression model.
          }\\
          \vspace{0.5em}

          The RMSE (Root Mean Square Error) is given by

          \begin{equation}\label{ex2-rmse}
              \text{RMSE} = \sqrt{\frac{1}{N} \sum_{i = 1}^{N} (z_i - \hat{z}_i)^2}
          \end{equation}

          Since we have 5 observations, $N = 5$.

          Using \eqref{ex1-final-z-hat} from the previous exercise, we can determine
          $\hat{z}$ for the 5 observations.
          Using matrix multiplication:

          $$
              \begin{aligned}
                  X            & = \begin{bmatrix}{}
                                       1.0000 & 0.8000 & 0.6400 & 0.5120 \\
                                       1.0000 & 1.0000 & 1.0000 & 1.0000 \\
                                       1.0000 & 1.2000 & 1.4400 & 1.7280 \\
                                       1.0000 & 1.4000 & 1.9600 & 2.7440 \\
                                       1.0000 & 1.6000 & 2.5600 & 4.0960
                                   \end{bmatrix} &
                  w            & = \begin{bmatrix}{}
                                       \textcolor{cblue}{7.0451}  \\
                                       \textcolor{cred}{4.6409}   \\
                                       \textcolor{cgreen}{1.9673} \\
                                       \textcolor{cyellow}{-1.3009}
                                   \end{bmatrix}       \\
                  \hat{z} = Xw & = \begin{bmatrix}{}
                                       11.3509 \\
                                       12.3525 \\
                                       13.1992 \\
                                       13.8287 \\
                                       14.1785
                                   \end{bmatrix}
                               &
                  z            & = \begin{bmatrix}
                                       24 \\ 20 \\ 10 \\ 13 \\ 12
                                   \end{bmatrix}
              \end{aligned}
          $$

          We can now calculate RMSE, using \eqref{ex2-rmse},

          $$
              \begin{aligned}
                  \sum_{i = 1}^{N} (z_i - \hat{z}_i)^2 & = 234.153                           \\
                  \text{RMSE}                          & = \sqrt{\frac{1}{5} \times 234.153} \\
                                                       & = 6.8433
              \end{aligned}
          $$

          Therefore, the RMSE is \textbf{6.8433}.

    \item {\color{questioncolor}\bfseries
          Consider a multi-layer perceptron characterized by one hidden layer with 2 nodes.
          Using the activation function $f(x) = e^{0.1 x}$ on all units, all weights
          initialized as 1 (including biases), and the half squared error loss,
          perform one batch gradient descent update (with learning rate $\eta = 0.1$)
          for the first three observations $(0.8)$, $(1)$ and $(1.2)$.
          }\\
          \vspace{0.5em}

          We start by drawing the multi-layer perceptron, where Layer 1 is the hidden layer.
          Since we only have one input and one output variable, Layers 0 and 2 only
          have a single node.

          \begin{figure}[H]
              \centering
              \includesvg[width=7cm]{assets/hw3-multi-layer-perceptron.svg}
              \caption{Multi-Layer Perceptron}
              \label{fig:multi-layer-perceptron}
          \end{figure}

          All weights and biases are initialized as

          $$
              \begin{aligned}
                  W^{[1]} & = \begin{bmatrix}
                                  1 \\
                                  1
                              \end{bmatrix} &
                  b^{[1]} & = \begin{bmatrix}
                                  1 \\
                                  1
                              \end{bmatrix}  \\
                  W^{[2]} & = \begin{bmatrix}
                                  1 & 1 \\
                              \end{bmatrix} &
                  b^{[2]} & = \begin{bmatrix}
                                  1
                              \end{bmatrix}
              \end{aligned}
          $$

          We can calculate the values of each layer, $x^{[i]}$, where $i$ is the
          index of the layer, by

          $$
              \begin{array}{c}
                  z^{[i]} = W^{[i]} \cdot x^{[i-1]} + b^{[i]} \\
                  x^{[i]} = f(z^{[i]})
              \end{array}
          $$

          where $f$ is the activation function, $f(x) = e^{0.1x}$ and $i > 0$.

          The error function, half squared error loss, is given by:

          $$
              E(x, t) = \cfrac{1}{2}(x - t)^2
          $$

          \vspace*{0.5cm}

          We can now calculate the values of each node in the multi-layer perceptron
          with the initialized weights and biases, for each of the three observations,
          that is, do \textbf{forward propagation}.

          \vspace*{0.5cm}

          \begin{paracol}{3}
              \begin{center}
                  \textbf{Observation: $(0.8)$}
              \end{center}

              $$
                  x^{[0]} = \begin{bmatrix}
                      0.8
                  \end{bmatrix}
              $$

              $$
                  \begin{aligned}
                      z^{[1]} & = W^{[1]} \cdot x^{[0]} + b^{[1]} \\
                              & = \begin{bmatrix}
                                      1 \\
                                      1
                                  \end{bmatrix}
                      \begin{bmatrix}
                          0.8
                      \end{bmatrix}
                      +
                      \begin{bmatrix}
                          1 \\
                          1
                      \end{bmatrix}                              \\
                              & = \begin{bmatrix}{}
                                      1.8 \\
                                      1.8
                                  \end{bmatrix}
                  \end{aligned}
              $$

              $$
                  \begin{aligned}
                      x^{[1]} & = f(z^{[1]})                  \\
                              & = f\left(\begin{bmatrix}{}
                                                 1.8 \\
                                                 1.8
                                             \end{bmatrix}\right) \\
                              & = \begin{bmatrix}{}
                                      1.1972 \\
                                      1.1972
                                  \end{bmatrix}
                  \end{aligned}
              $$

              $$
                  \begin{aligned}
                      z^{[2]} & = W^{[2]} \cdot x^{[1]} + b^{[2]} \\
                              & = \begin{bmatrix}
                                      1 &
                                      1
                                  \end{bmatrix}
                      \begin{bmatrix}
                          1.1972 \\
                          1.1972
                      \end{bmatrix}
                      +
                      \begin{bmatrix}
                          1
                      \end{bmatrix}                              \\
                              & = \begin{bmatrix}{}
                                      3.3944
                                  \end{bmatrix}
                  \end{aligned}
              $$

              $$
                  \begin{aligned}
                      x^{[2]} & = f(z^{[2]})                  \\
                              & = f\left(\begin{bmatrix}{}
                                                 3.3944
                                             \end{bmatrix}\right) \\
                              & = \begin{bmatrix}{}
                                      1.4042
                                  \end{bmatrix}
                  \end{aligned}
              $$

              \switchcolumn

              \begin{center}
                  \textbf{Observation: $(1.0)$}
              \end{center}

              $$
                  x^{[0]} = \begin{bmatrix}
                      1.0
                  \end{bmatrix}
              $$

              $$
                  \begin{aligned}
                      z^{[1]} & = W^{[1]} \cdot x^{[0]} + b^{[1]} \\
                              & = \begin{bmatrix}
                                      1 \\
                                      1
                                  \end{bmatrix}
                      \begin{bmatrix}
                          1.0
                      \end{bmatrix}
                      +
                      \begin{bmatrix}
                          1 \\
                          1
                      \end{bmatrix}                              \\
                              & = \begin{bmatrix}{}
                                      2 \\
                                      2
                                  \end{bmatrix}
                  \end{aligned}
              $$

              $$
                  \begin{aligned}
                      x^{[1]} & = f(z^{[1]})                  \\
                              & = f\left(\begin{bmatrix}{}
                                                 2 \\
                                                 2
                                             \end{bmatrix}\right) \\
                              & = \begin{bmatrix}{}
                                      1.2214 \\
                                      1.2214
                                  \end{bmatrix}
                  \end{aligned}
              $$

              $$
                  \begin{aligned}
                      z^{[2]} & = W^{[2]} \cdot x^{[1]} + b^{[2]} \\
                              & = \begin{bmatrix}
                                      1 &
                                      1
                                  \end{bmatrix}
                      \begin{bmatrix}
                          1.2214 \\
                          1.2214
                      \end{bmatrix}
                      +
                      \begin{bmatrix}
                          1
                      \end{bmatrix}                              \\
                              & = \begin{bmatrix}{}
                                      3.4428
                                  \end{bmatrix}
                  \end{aligned}
              $$

              $$
                  \begin{aligned}
                      x^{[2]} & = f(z^{[2]})                  \\
                              & = f\left(\begin{bmatrix}{}
                                                 3.4428
                                             \end{bmatrix}\right) \\
                              & = \begin{bmatrix}{}
                                      1.4110
                                  \end{bmatrix}
                  \end{aligned}
              $$

              \switchcolumn

              \begin{center}
                  \textbf{Observation: $(1.2)$}
              \end{center}

              $$
                  x^{[0]} = \begin{bmatrix}
                      1.2
                  \end{bmatrix}
              $$

              $$
                  \begin{aligned}
                      z^{[1]} & = W^{[1]} \cdot x^{[0]} + b^{[1]} \\
                              & = \begin{bmatrix}
                                      1 \\
                                      1
                                  \end{bmatrix}
                      \begin{bmatrix}
                          1.2
                      \end{bmatrix}
                      +
                      \begin{bmatrix}
                          1 \\
                          1
                      \end{bmatrix}                              \\
                              & = \begin{bmatrix}{}
                                      2.2 \\
                                      2.2
                                  \end{bmatrix}
                  \end{aligned}
              $$

              $$
                  \begin{aligned}
                      x^{[1]} & = f(z^{[1]})                  \\
                              & = f\left(\begin{bmatrix}{}
                                                 2.2 \\
                                                 2.2
                                             \end{bmatrix}\right) \\
                              & = \begin{bmatrix}{}
                                      1.2461 \\
                                      1.2461
                                  \end{bmatrix}
                  \end{aligned}
              $$

              $$
                  \begin{aligned}
                      z^{[2]} & = W^{[2]} \cdot x^{[1]} + b^{[2]} \\
                              & = \begin{bmatrix}
                                      1 &
                                      1
                                  \end{bmatrix}
                      \begin{bmatrix}
                          1.2461 \\
                          1.2461
                      \end{bmatrix}
                      +
                      \begin{bmatrix}
                          1
                      \end{bmatrix}                              \\
                              & = \begin{bmatrix}{}
                                      3.4922
                                  \end{bmatrix}
                  \end{aligned}
              $$

              $$
                  \begin{aligned}
                      x^{[2]} & = f(z^{[2]})                  \\
                              & = f\left(\begin{bmatrix}{}
                                                 3.4922
                                             \end{bmatrix}\right) \\
                              & = \begin{bmatrix}{}
                                      1.4180
                                  \end{bmatrix}
                  \end{aligned}
              $$

          \end{paracol}

          Now that we have the output for each of our observations, considering
          the initial weights and biases, we can apply \textbf{back propagation} to
          perform a batch gradient descent update.

          We want to batch update the weights ($W^{[1]}$ and $W^{[2]}$) and biases
          ($b^{[1]}$ and $b^{[2]}$). The new, updated values, are given by:

          $$
              \Delta W^{[i]} = - \eta \frac{\partial E}{\partial W^{[i]}}
              \quad,\quad
              \Delta b^{[i]} = - \eta \frac{\partial E}{\partial b^{[i]}}
          $$

          Applying the chain rule:

          $$
              \frac{\partial E}{\partial W^{[2]}} =
              \underbrace{\frac{\partial E}{\partial x^{[2]}}
                  \times
                  \frac{\partial x^{[2]}}{\partial z^{[2]}}
              }_{\delta^{[2]}}
              \times
              \frac{\partial z^{[2]}}{\partial W^{[2]}}
              \quad,\quad
              \frac{\partial E}{\partial b^{[2]}} =
              \underbrace{\frac{\partial E}{\partial x^{[2]}}
                  \times
                  \frac{\partial x^{[2]}}{\partial z^{[2]}}
              }_{\delta^{[2]}}
              \times
              \frac{\partial z^{[2]}}{\partial b^{[2]}}
          $$
          $$
              \frac{\partial E}{\partial x^{[2]}}
              = \frac{1}{2} \left( 2x^{[2]} - 2t \right)
              = x^{[2]} - t
          $$
          $$
              \frac{\partial x^{[i]}}{\partial z^{[i]}} = f'\left(z^{[i]}\right) = 0.1 e^{0.1 z^{[i]}}
          $$
          $$
              \frac{\partial z^{[i]}}{\partial W^{[i]}} = x^{[i-1]}
              \quad,\quad
              \frac{\partial z^{[i]}}{\partial b^{[i]}} = 1
          $$

          Therefore:

          \begin{equation}\label{ex3-vars-layer-2}
              \delta^{[2]} = (x^{[2]} - t) \circ 0.1 e^{0.1 z^{[2]}}
              \quad,\quad
              \Delta W^{[2]} = - 0.1 \delta^{[2]} \left(x^{[1]}\right)^T
              \quad,\quad
              \Delta b^{[2]} = - 0.1 \delta^{[2]}
          \end{equation}

          We can repeat this process for layer 1, yielding:

          $$
              \frac{\partial E}{\partial W^{[1]}} = \delta^{[1]} \frac{\partial z^{[1]}}{\partial W^{[1]}}
              = \delta^{[1]} \left(x^{[0]}\right)^T
          $$
          \begin{equation}\label{ex3-vars-layer-1}
              \delta^{[1]} = \left(W^{[2]}\right)^T \delta^{[2]} \circ 0.1 e^{0.1 z^{[1]}}
              \quad,\quad
              \Delta W^{[1]} = - 0.1 \delta^{[1]} \left(x^{[0]}\right)^T
              \quad,\quad
              \Delta b^{[1]} = - 0.1 \delta^{[1]}
          \end{equation}

          Calculating $\Delta W^{[2]}$, $\Delta W^{[1]}$, $\Delta b^{[2]}$ and $\Delta b^{[1]}$
          for each observation:

          \begin{paracol}{3}
              \begin{center}
                  \textbf{Observation: $(0.8)$}
              \end{center}

              Remembering the target, $t = 24$,
              the output, $x^{[2]} = 1.4042$,
              and $z^{[2]} = 3.3944$,
              calculate $\delta^{[2]}$ using \eqref{ex3-vars-layer-2}:

              \begin{footnotesize}
                  $$
                      \begin{aligned}
                          \delta^{[2]} & = \left(x^{[2]} - t\right) \times 0.1 e^{0.1 z^{[2]}} \\
                                       & = (1.4042 - 24) \times 0.1 e^{0.1 \times 3.3944}      \\
                                       & = -3.1728
                      \end{aligned}
                  $$
              \end{footnotesize}

              Using \eqref{ex3-vars-layer-1}, calculate $\delta^{[1]}$,
              \begin{footnotesize}
                  $$
                      \begin{aligned}
                          \delta^{[1]} & = \left(W^{[2]}\right)^T \delta^{[2]} \circ 0.1 e^{0.1 z^{[1]}} \\
                                       & = \begin{bmatrix}
                                               1 \\ 1
                                           \end{bmatrix}
                          \begin{bmatrix}
                              -3.1728
                          \end{bmatrix}
                          \circ
                          0.1 e^{0.1 \begin{bmatrix}
                                                 1.8 \\
                                                 1.8
                                             \end{bmatrix}}                                                      \\
                                       & = \begin{bmatrix}
                                               -0.3799 \\
                                               -0.3799
                                           \end{bmatrix}
                      \end{aligned}
                  $$
              \end{footnotesize}

              Finally, using \eqref{ex3-vars-layer-2} and \eqref{ex3-vars-layer-1},
              calculate:

              \begin{footnotesize}
                  $$
                      \begin{aligned}
                          \Delta W^{[2]} & = -0.1 \delta^{[2]} \left(x^{[1]}\right)^T \\
                                         & = -0.1 \times (-3.1728) \begin{bmatrix}
                                                                       1.1972 \\
                                                                       1.1972
                                                                   \end{bmatrix}^T    \\
                                         & = \textcolor{cblue}{
                              \begin{bmatrix}
                                  0.3799 & 0.3799
                              \end{bmatrix}
                          }
                      \end{aligned}
                  $$

                  $$
                      \begin{aligned}
                          \Delta b^{[2]} & = -0.1 \delta^{[2]}        \\
                                         & = -0.1 \times (-3.1728)    \\
                                         & = \textcolor{cred}{0.3173}
                      \end{aligned}
                  $$

                  $$
                      \begin{aligned}
                          \Delta W^{[1]} & = -0.1 \delta^{[1]} \left(x^{[0]}\right)^T \\
                                         & = -0.1 \times \begin{bmatrix}
                                                             -0.3799 \\
                                                             -0.3799
                                                         \end{bmatrix} \begin{bmatrix}
                                                                           0.8
                                                                       \end{bmatrix} \\
                                         & = \textcolor{cgreen}{
                              \begin{bmatrix}
                                  0.0304 \\
                                  0.0304
                              \end{bmatrix}
                          }
                      \end{aligned}
                  $$

                  $$
                      \begin{aligned}
                          \Delta b^{[1]} & = -0.1 \delta^{[1]}    \\
                                         & = -0.1 \begin{bmatrix}
                                                      -0.3799 \\
                                                      -0.3799
                                                  \end{bmatrix}  \\
                                         & = \textcolor{cyellow}{
                              \begin{bmatrix}
                                  0.0380 \\
                                  0.0380
                              \end{bmatrix}
                          }
                      \end{aligned}
                  $$
              \end{footnotesize}

              \switchcolumn

              \begin{center}
                  \textbf{Observation: $(1.0)$}
              \end{center}

              Remembering the target, $t = 20$,
              the output, $x^{[2]} = 1.4110$,
              and $z^{[2]} = 3.4428$,
              calculate $\delta^{[2]}$ using \eqref{ex3-vars-layer-2}:

              \begin{footnotesize}
                  $$
                      \begin{aligned}
                          \delta^{[2]} & = \left(x^{[2]} - t\right) \times 0.1 e^{0.1 z^{[2]}} \\
                                       & = (1.4110 - 20) \times 0.1 e^{0.1 \times 3.4428}      \\
                                       & = -2.6229
                      \end{aligned}
                  $$
              \end{footnotesize}

              Using \eqref{ex3-vars-layer-1}, calculate $\delta^{[1]}$,
              \begin{footnotesize}
                  $$
                      \begin{aligned}
                          \delta^{[1]} & = \left(W^{[2]}\right)^T \delta^{[2]} \circ 0.1 e^{0.1 z^{[1]}} \\
                                       & = \begin{bmatrix}
                                               1 \\ 1
                                           \end{bmatrix}
                          \begin{bmatrix}
                              -2.6229
                          \end{bmatrix}
                          \circ
                          0.1 e^{0.1 \begin{bmatrix}
                                                 2 \\
                                                 2
                                             \end{bmatrix}}                                                      \\
                                       & = \begin{bmatrix}
                                               -0.3204 \\
                                               -0.3204
                                           \end{bmatrix}
                      \end{aligned}
                  $$
              \end{footnotesize}

              Finally, using \eqref{ex3-vars-layer-2} and \eqref{ex3-vars-layer-1},
              calculate:

              \begin{footnotesize}
                  $$
                      \begin{aligned}
                          \Delta W^{[2]} & = -0.1 \delta^{[2]} \left(x^{[1]}\right)^T \\
                                         & = -0.1 \times (-2.6229) \begin{bmatrix}
                                                                       1.2214 \\
                                                                       1.2214
                                                                   \end{bmatrix}^T    \\
                                         & = \textcolor{cblue}{
                              \begin{bmatrix}
                                  0.3204 & 0.3204
                              \end{bmatrix}
                          }
                      \end{aligned}
                  $$

                  $$
                      \begin{aligned}
                          \Delta b^{[2]} & = -0.1 \delta^{[2]}        \\
                                         & = -0.1 \times (-2.6229)    \\
                                         & = \textcolor{cred}{0.2623}
                      \end{aligned}
                  $$

                  $$
                      \begin{aligned}
                          \Delta W^{[1]} & = -0.1 \delta^{[1]} \left(x^{[0]}\right)^T \\
                                         & = -0.1 \times \begin{bmatrix}
                                                             -0.3204 \\
                                                             -0.3204
                                                         \end{bmatrix} \begin{bmatrix}
                                                                           1
                                                                       \end{bmatrix} \\
                                         & = \textcolor{cgreen}{
                              \begin{bmatrix}
                                  0.0320 \\
                                  0.0320
                              \end{bmatrix}
                          }
                      \end{aligned}
                  $$

                  $$
                      \begin{aligned}
                          \Delta b^{[1]} & = -0.1 \delta^{[1]}    \\
                                         & = -0.1 \begin{bmatrix}
                                                      -0.3204 \\
                                                      -0.3204
                                                  \end{bmatrix}  \\
                                         & = \textcolor{cyellow}{
                              \begin{bmatrix}
                                  0.0320 \\
                                  0.0320
                              \end{bmatrix}
                          }
                      \end{aligned}
                  $$
              \end{footnotesize}

              \switchcolumn

              \begin{center}
                  \textbf{Observation: $(1.2)$}
              \end{center}

              Remembering the target, $t = 10$,
              the output, $x^{[2]} = 1.4180$,
              and $z^{[2]} = 3.4922$,
              calculate $\delta^{[2]}$ using \eqref{ex3-vars-layer-2}:

              \begin{footnotesize}
                  $$
                      \begin{aligned}
                          \delta^{[2]} & = \left(x^{[2]} - t\right) \times 0.1 e^{0.1 z^{[2]}} \\
                                       & = (1.4180 - 10) \times 0.1 e^{0.1 \times 3.4922}      \\
                                       & = -1.2169
                      \end{aligned}
                  $$
              \end{footnotesize}

              Using \eqref{ex3-vars-layer-1}, calculate $\delta^{[1]}$,
              \begin{footnotesize}
                  $$
                      \begin{aligned}
                          \delta^{[1]} & = \left(W^{[2]}\right)^T \delta^{[2]} \circ 0.1 e^{0.1 z^{[1]}} \\
                                       & = \begin{bmatrix}
                                               1 \\ 1
                                           \end{bmatrix}
                          \begin{bmatrix}
                              -1.2169
                          \end{bmatrix}
                          \circ
                          0.1 e^{0.1 \begin{bmatrix}
                                                 2.2 \\
                                                 2.2
                                             \end{bmatrix}}                                                      \\
                                       & = \begin{bmatrix}
                                               -0.1516 \\
                                               -0.1516
                                           \end{bmatrix}
                      \end{aligned}
                  $$
              \end{footnotesize}

              Finally, using \eqref{ex3-vars-layer-2} and \eqref{ex3-vars-layer-1},
              calculate:

              \begin{footnotesize}
                  $$
                      \begin{aligned}
                          \Delta W^{[2]} & = -0.1 \delta^{[2]} \left(x^{[1]}\right)^T \\
                                         & = -0.1 \times (-1.2169) \begin{bmatrix}
                                                                       1.2461 \\
                                                                       1.2461
                                                                   \end{bmatrix}^T    \\
                                         & = \textcolor{cblue}{
                              \begin{bmatrix}
                                  0.1516 & 0.1516
                              \end{bmatrix}
                          }
                      \end{aligned}
                  $$

                  $$
                      \begin{aligned}
                          \Delta b^{[2]} & = -0.1 \delta^{[2]}        \\
                                         & = -0.1 \times (-1.2169)    \\
                                         & = \textcolor{cred}{0.1217}
                      \end{aligned}
                  $$

                  $$
                      \begin{aligned}
                          \Delta W^{[1]} & = -0.1 \delta^{[1]} \left(x^{[0]}\right)^T \\
                                         & = -0.1 \times \begin{bmatrix}
                                                             -0.1516 \\
                                                             -0.1516
                                                         \end{bmatrix} \begin{bmatrix}
                                                                           1.2
                                                                       \end{bmatrix} \\
                                         & = \textcolor{cgreen}{
                              \begin{bmatrix}
                                  0.0182 \\
                                  0.0182
                              \end{bmatrix}
                          }
                      \end{aligned}
                  $$

                  $$
                      \begin{aligned}
                          \Delta b^{[1]} & = -0.1 \delta^{[1]}    \\
                                         & = -0.1 \begin{bmatrix}
                                                      -0.1516 \\
                                                      -0.1516
                                                  \end{bmatrix}  \\
                                         & = \textcolor{cyellow}{
                              \begin{bmatrix}
                                  0.0152 \\
                                  0.0152
                              \end{bmatrix}
                          }
                      \end{aligned}
                  $$
              \end{footnotesize}

          \end{paracol}

          Finally, we can \textbf{update the weights and biases}:

          $$
              \begin{aligned}
                  W^{[2]}_{\text{new}} & = W^{[2]}_{\text{old}} + \sum_{i=1}^{3} \textcolor{cblue}{\Delta W^{[2]}_i}   \\
                                       & = \begin{bmatrix}
                                               1 & 1
                                           \end{bmatrix} + \left(
                  \textcolor{cblue}{
                      \begin{bmatrix}
                          0.3799 & 0.3799
                      \end{bmatrix}
                  }
                  +
                  \textcolor{cblue}{
                      \begin{bmatrix}
                          0.3204 & 0.3204
                      \end{bmatrix}
                  }
                  +
                  \textcolor{cblue}{
                      \begin{bmatrix}
                          0.1516 & 0.1516
                      \end{bmatrix}
                  }
                  \right)                                                                                              \\
                                       & = \begin{bmatrix}
                                               1.8518 & 1.8518
                                           \end{bmatrix}
                  \\
                  \\
                  b^{[2]}_{\text{new}} & = b^{[2]}_{\text{old}} + \sum_{i=1}^{3} \textcolor{cred}{\Delta b^{[2]}_i}    \\
                                       & = \begin{bmatrix}
                                               1
                                           \end{bmatrix} + \left(
                  \textcolor{cred}{
                      \begin{bmatrix}
                          0.3173
                      \end{bmatrix}
                  }
                  +
                  \textcolor{cred}{
                      \begin{bmatrix}
                          0.2623
                      \end{bmatrix}
                  }
                  +
                  \textcolor{cred}{
                      \begin{bmatrix}
                          0.1217
                      \end{bmatrix}
                  }
                  \right)                                                                                              \\
                                       & = \begin{bmatrix}
                                               1.7013
                                           \end{bmatrix}
                  \\
                  \\
                  W^{[1]}_{\text{new}} & = W^{[1]}_{\text{old}} + \sum_{i=1}^{3} \textcolor{cgreen}{\Delta W^{[1]}_i}  \\
                                       & = \begin{bmatrix}
                                               1 \\
                                               1
                                           \end{bmatrix} + \left(
                  \textcolor{cgreen}{
                      \begin{bmatrix}
                          0.0304 \\
                          0.0304
                      \end{bmatrix}
                  }
                  +
                  \textcolor{cgreen}{
                      \begin{bmatrix}
                          0.0320 \\
                          0.0320
                      \end{bmatrix}
                  }
                  +
                  \textcolor{cgreen}{
                      \begin{bmatrix}
                          0.0182 \\
                          0.0182
                      \end{bmatrix}
                  }
                  \right)                                                                                              \\
                                       & = \begin{bmatrix}
                                               1.0806 \\
                                               1.0806
                                           \end{bmatrix}
                  \\
                  \\
                  b^{[1]}_{\text{new}} & = b^{[1]}_{\text{old}} + \sum_{i=1}^{3} \textcolor{cyellow}{\Delta b^{[1]}_i} \\
                                       & = \begin{bmatrix}
                                               1 \\
                                               1
                                           \end{bmatrix} + \left(
                  \textcolor{cyellow}{
                      \begin{bmatrix}
                          0.0380 \\
                          0.0380
                      \end{bmatrix}
                  }
                  +
                  \textcolor{cyellow}{
                      \begin{bmatrix}
                          0.0320 \\
                          0.0320
                      \end{bmatrix}
                  }
                  +
                  \textcolor{cyellow}{
                      \begin{bmatrix}
                          0.0152 \\
                          0.0152
                      \end{bmatrix}
                  }
                  \right)                                                                                              \\
                                       & = \begin{bmatrix}
                                               1.0852 \\
                                               1.0852
                                           \end{bmatrix}
              \end{aligned}
          $$

          To conclude, the weights and biases after one batch gradient descent update,
          with learning rate $\eta = 0.1$, are,

          $$
              \begin{aligned}
                  W^{[1]} & = \begin{bmatrix}
                                  1.0806 \\
                                  1.0806
                              \end{bmatrix} &
                  b^{[1]} & = \begin{bmatrix}
                                  1.0852 \\
                                  1.0852
                              \end{bmatrix}  \\
                  W^{[2]} & = \begin{bmatrix}
                                  1.8518 & 1.8518
                              \end{bmatrix} &
                  b^{[2]} & = \begin{bmatrix}
                                  1.7013
                              \end{bmatrix}
              \end{aligned}
          $$
\end{enumerate}

\pagebreak

\begin{center}
    \large{\textbf{Part II}: Programming and critical analysis}
\end{center}

{\color{questioncolor}\bfseries
\noindent
Consider the following three regressors applied on \texttt{kin8nm.arff} data
(available at the webpage):
\begin{itemize}
    \item linear regression with Ridge regularization term of 0.1
    \item two MLPs - $MLP_1$ and $MLP_2$ - each with two hidden layers of
          size 10, hyperbolic tagent function as the activation function of all nodes,
          a maximum of 500 iterations, and a fixed seed (\texttt{random\_state=0}).
          $MLP_1$ should be parameterized with early stopping while $MLP_2$ should not
          consider early stopping.
          Remaining parameters (e.g., loss function, batch size, regularization term,
          solver) should be set as default.
\end{itemize}
Using a 70-30 training-test split with a fixed seed (\texttt{random\_state=0}):
}

\begin{enumerate}[leftmargin=\labelsep,resume]
    \item {\color{questioncolor}\bfseries
          Compute the MAE of the three regressors: linear regression, $MLP_1$ and $MLP_2$.
          }\\
          \vspace{0.5em}

          We start by applying the three regressors, as said in the question prompt,
          and then calculate the Mean Absolute Error by using \texttt{sklearn}'s
          \texttt{metrics.mean\_absolute\_error}.

          The respective code can be found on listing \ref{listing-ex4}.

          \begin{center}
              \captionsetup{type=table}
              \begin{tabular}{c|c|c|c}
                                      & Ridge regularization & $MLP_1$ & $MLP_2$ \\
                  \hline
                  Mean Absolute Error & 0.1628               & 0.0680  & 0.0978
              \end{tabular}
              \captionof{table}{Mean Absolute Error for each of the three regressors}
          \end{center}


    \item {\color{questioncolor}\bfseries
          Plot the residues (in absolute value) using two visualizations: boxplots
          and histograms.\\
          \textit{Hint}: consider using \texttt{boxplot} and \texttt{hist} functions
          from \texttt{matplotlib.pyplot} to this end.
          }\\
          \vspace{0.5em}

          The residues of each regressor can be calculated as
          the absolute difference between the expected value and the
          predicted value, for each observation.

          The respective code can be found on listing \ref{listing-ex5}.

          \begin{figure}[H]
              \centering
              \includesvg[width = 12cm]{assets/hw3-boxplot.svg}
              \caption{Boxplot of the residues of the three regressors}
              \label{fig:bloxplot}
          \end{figure}

          \begin{figure}[H]
              \centering
              \includesvg[width = 12cm]{assets/hw3-histogram.svg}
              \caption{Histogram of the residues of the three regressors}
              \label{fig:histogram}
          \end{figure}

    \item {\color{questioncolor}\bfseries
          How many iterations were required for $MLP_1$ and $MLP_2$ to converge?
          }\\
          \vspace{0.5em}

          It is possible to get the number of iterations through the
          \texttt{n\_iter\_} property of the \texttt{MLPRegressor} object.

          The respective code can be found on listing \ref{listing-ex6}.

          \begin{center}
              \captionsetup{type=table}
              \begin{tabular}{c|c|c}
                                       & $MLP_1$ & $MLP_2$ \\
                  \hline
                  Number of Iterations & 452     & 77
              \end{tabular}
              \captionof{table}{Number of iterations for each of the two MLP regressors}
          \end{center}

    \item {\color{questioncolor}\bfseries
          What can be motivating the unexpected differences on the number of iterations?
          Hypothesize one reason underlying the observed performance differences between the MLPs.
          }\\
          \vspace{0.5em}

          By comparing the number of iteration of both models, we can conclude that
          $MLP1$ takes a lot more iterations ($452 > 77$) to converge than $MLP2$.
          This is because $MLP1$ uses early stopping, which means that the model
          stops training when the validation score is not improving anymore.
          By reading the documentation about the
          \href{https://scikit-learn.org/stable/modules/generated/sklearn.neural_network.MLPRegressor.html}{\texttt{sklearn.neural\_network.MLPRegressor}},
          we can see that the default value for the validation fraction is
          \texttt{validation\_fraction} (default value = 0.1), which means that 10\%
          of the training data is used for validation and the other 90\% data
          to train the model.
          After each iteration, the model will calculate the validation score.
          If the score is not improving by at least \texttt{tol} (default value = 1e-4)
          for \texttt{n\_iter\_no\_change} (default value = 10) iterations,
          the model will stop training.
          The $MLP2$, by contrast, will stop when it minimizes the loss function.
          This allows the $MLP1$, when applied to this dataset, to have better performance
          (lower MAE value, and lower residue values) than $MLP2$, but it takes a
          lot more iterations to converge.

          Since the $MLP1$ uses early stopping and $MLP2$ does not, we can
          attribute the difference in performance to the early stopping.
          Due to how the early stopping works, similar to splitting the data
          into training and testing, early stopping splits the training data
          into training and validation sets.
          While training the model, the training set will be used for fitting
          while the validation set will be used for calculating the validation
          score of the model.
          This will cause the model to learn more effectively, which will cause
          the model to be more robust and generalize better, especially by
          \textbf{avoiding overfitting}.
          This is why the $MLP1$ has a lower MAE value and lower residue values than $MLP2$,
          and thus better performance.

\end{enumerate}

\pagebreak

\center\large{\textbf{Appendix}\vskip 0.3cm}

\lstinputlisting[label={listing-ex4},caption={Train, test and compute the MAE of the three regressors},language=Python]{assets/hw3-code-4.py}

\vspace*{1.5cm}

\lstinputlisting[label={listing-ex5},caption={Plot the residues using a bloxplot and a histogram},language=Python]{assets/hw3-code-5.py}

\vspace*{1.5cm}

\lstinputlisting[label={listing-ex6},caption={Print the number of iterations to converge of the regressors},language=Python]{assets/hw3-code-6.py}

\end{document}

\documentclass[12pt]{article}
\usepackage[paper=letterpaper,margin=2cm]{geometry}
\usepackage{amsmath}
\usepackage{amssymb}
\usepackage{amsfonts}
\usepackage{newtxtext, newtxmath}
\usepackage{enumitem}
\usepackage{titling}
\usepackage[colorlinks=true]{hyperref}
\usepackage{multirow}
\usepackage{svg}
\usepackage{listings}
\usepackage{xcolor}
\usepackage{float}
\usepackage{graphicx}
\usepackage[font=footnotesize]{caption}
\usepackage{subcaption}
\usepackage{paracol}
\usepackage{changepage}

\setlength{\droptitle}{-6em}

\definecolor{codegreen}{rgb}{0,0.6,0}
\definecolor{codegray}{rgb}{0.5,0.5,0.5}
\definecolor{codepurple}{rgb}{0.58,0,0.82}
\definecolor{backcolour}{rgb}{0.95,0.95,0.92}

% TEXT COLORS
% https://coolors.co/4c6085-1588e0-0cac8c-f25f5c-b89300
\definecolor{cgray}{HTML}{4c6085}
\definecolor{cblue}{HTML}{1588e0}
\definecolor{cgreen}{HTML}{0cac8c}
\definecolor{cred}{HTML}{f25f5c}
\definecolor{cyellow}{HTML}{b89300}
\definecolor{corange}{HTML}{f58b00}

% BACKGROUND COLORS
% https://coolors.co/8a9cbc-7cc0f3-66f4d8-f6908e-ffe270
\definecolor{bgray}{HTML}{8a9cbc}
\definecolor{bblue}{HTML}{7cc0f3}
\definecolor{bgreen}{HTML}{66f4d8}
\definecolor{bred}{HTML}{f6908e}
\definecolor{byellow}{HTML}{ffe270}
\definecolor{borange}{HTML}{ffca85}

\definecolor{linkcolor}{HTML}{f57429}
\definecolor{questioncolor}{HTML}{444444}
\definecolor{listingbackground}{HTML}{f5f5f5}

\hypersetup{
    allcolors=linkcolor
}

\setlength{\columnseprule}{1pt}
\def\columnseprulecolor{\color{questioncolor}}

\lstdefinestyle{mystyle}{
    commentstyle=\color{codegreen},
    keywordstyle=\color{magenta},
    numberstyle=\tiny\color{codegray},
    stringstyle=\color{codepurple},
    basicstyle=\ttfamily\footnotesize,
    backgroundcolor=\color{listingbackground},
    breakatwhitespace=false,
    breaklines=true,
    captionpos=b,
    keepspaces=true,
    numbers=left,
    numbersep=5pt,
    showspaces=false,
    showstringspaces=false,
    showtabs=false,
    tabsize=2
}

\lstset{style=mystyle}

\title{\large{Aprendizagem 2022}\vskip 0.2cm Homework IV -- Group 020}
\date{}
\author{Diogo Correia (99211) \and Tomás Esteves (99341)}
\begin{document}
\maketitle
\begin{center}
    \large{\vskip -1.0cm\textbf{Part I}: Pen and paper}
\end{center}

{
\color{questioncolor}\bfseries
\noindent
Given the bivariate observations $
    \left\{
    \begin{pmatrix}
        1 \\
        2
    \end{pmatrix},
    \begin{pmatrix}
        -1 \\
        1
    \end{pmatrix},
    \begin{pmatrix}
        1 \\
        0
    \end{pmatrix}
    \right\}
$, and the multivariate Gaussian mixture
$$
    u_1 = \begin{pmatrix}
        2 \\
        2
    \end{pmatrix}
    \quad,\quad
    u_2 = \begin{pmatrix}
        0 \\
        0
    \end{pmatrix}
    \quad,\quad
    \varSigma_1 = \begin{pmatrix}
        2 & 1 \\
        1 & 2
    \end{pmatrix}
    \quad,\quad
    \varSigma_2 = \begin{pmatrix}
        2 & 0 \\
        0 & 2
    \end{pmatrix}
    \quad,\quad
    \pi_1 = 0.5
    \quad,\quad
    \pi_2 = 0.5
$$
}

\begin{enumerate}[leftmargin=\labelsep]
    \item {\color{questioncolor}\bfseries
          Perform one epoch of the EM clustering algorithm and determine the new parameters.
          Indicate all calculus step by step (you can use a computer, however disclose
          intermediary steps).
          }\\
          \vspace{0.5em}

          The EM (Expectation-Maximization) algorithm has four major steps:
          Initialization, Expectation, Maximization and Evaluate.

          \textbf{Initialization}

          We'll start by labeling each observation,

          $$
              \textcolor{cred}{x_1 = \begin{pmatrix}
                      1 \\
                      2
                  \end{pmatrix}}
              \quad,\quad
              \textcolor{cblue}{x_2 = \begin{pmatrix}
                      -1 \\
                      1
                  \end{pmatrix}}
              \quad,\quad
              \textcolor{corange}{x_3 = \begin{pmatrix}
                      1 \\
                      0
                  \end{pmatrix}}
          $$

          and considering the initial parameters, $u_1$, $u_2$, $\varSigma_1$,
          $\varSigma_2$, $\pi_1$ and $\pi_2$ as given by the question prompt,


          \begin{center}
              \captionsetup{type=table}
              \begin{tabular}{c|ccc}
                  Cluster                       & $u$ & $\varSigma$ & $\pi$ \\
                  \hline
                  \colorbox{bgreen}{Cluster 1}  &
                  $\begin{pmatrix}
                           2 \\
                           2
                       \end{pmatrix}$              &
                  $\begin{pmatrix}
                           2 & 1 \\
                           1 & 2
                       \end{pmatrix}$              &
                  0.5                                                       \\
                  \colorbox{byellow}{Cluster 2} &
                  $\begin{pmatrix}
                           0 \\
                           0
                       \end{pmatrix}$              &
                  $\begin{pmatrix}
                           2 & 0 \\
                           0 & 2
                       \end{pmatrix}$              &
                  0.5                                                       \\
              \end{tabular}
              \captionof{table}{Initial parameters for the 2 clusters in the multivariate Gaussian mixture}
              \label{ex1-initial-params-table}
          \end{center}

          \textbf{Expectation (E-step)}

          In this step we must assign each point to the cluster that yields
          the highest posterior probability.

          The posterior probability, $p(c_k | x_i)$, is given by

          \begin{equation}\label{ex1-posterior}
              \gamma_{ki} = p(c_k | x_i) = \frac{p(x_i|c_k)p(c_k)}{p(x_i)}
          \end{equation}

          We know the likelihood, $p(x_i|c_k)$, in the context of this multivariate Gaussian mixture, is given by

          \begin{equation}\label{ex1-likelihood}
              p(x_i|c_k) = \mathcal{N}(x_i|u_{k}, \Sigma_{k})
              = \frac{1}{2\pi \times |\varSigma|^{1/2}}
              \exp\left(-\frac{1}{2} \left(x_i - u_k\right)^T \varSigma^{-1}\left(x_i - u_k\right)\right)
          \end{equation}

          Then, it is possible to calculate the joint probability, $p(x_i, c_k)$,

          \begin{equation}\label{ex1-joint}
              p(x_i,c_k) = p(x_i|c_k)p(c_k) = p(x_i|c_k) \pi_k
          \end{equation}

          Finally, we can calculate the value of $p(x_i)$, which does not depend
          on the cluster, and is defined by the sum of $p(x_i, c_k)$, for each cluster,

          \begin{equation}\label{ex1-sum-joint}
              p(x_i) = \sum_k p(x_i, c_k)
          \end{equation}

          We now have all the building blocks to calculate the posterior probabilities
          for each combination of observation, $x_i$ and cluster, $c_k$.

          \vspace*{0.5cm}

          \begin{paracol}{3}
              \begin{center}
                  \textbf{Observation: \textcolor{cred}{$x_1 = \begin{pmatrix}
                                  1 \\
                                  2
                              \end{pmatrix}$}}
              \end{center}

              \switchcolumn

              \begin{center}
                  \textbf{Observation: \textcolor{cblue}{$x_2 = \begin{pmatrix}
                                  -1 \\
                                  1
                              \end{pmatrix}$}}
              \end{center}

              \switchcolumn

              \begin{center}
                  \textbf{Observation: \textcolor{corange}{$x_3 = \begin{pmatrix}
                                  1 \\
                                  0
                              \end{pmatrix}$}}
              \end{center}

          \end{paracol}

          \begin{center}
              We'll start by calculating the likelihood, using \eqref{ex1-likelihood},
              for each cluster,
          \end{center}

          \begin{paracol}{3}
              $$
                  \begin{aligned}
                      p(\textcolor{cred}{x_1}|\colorbox{bgreen}{$c=1$})  & = \mathcal{N}(\textcolor{cred}{x_1}|\colorbox{bgreen}{$u_{1}$}, \colorbox{bgreen}{$\Sigma_{1}$})   \\
                                                                         & = 0.06584                                                                                          \\
                      p(\textcolor{cred}{x_1}|\colorbox{byellow}{$c=2$}) & = \mathcal{N}(\textcolor{cred}{x_1}|\colorbox{byellow}{$u_{2}$}, \colorbox{byellow}{$\Sigma_{2}$}) \\
                                                                         & = 0.02280
                  \end{aligned}
              $$

              \switchcolumn

              $$
                  \begin{aligned}
                      p(\textcolor{cblue}{x_2}|\colorbox{bgreen}{$c=1$})  & = \mathcal{N}(\textcolor{cblue}{x_2}|\colorbox{bgreen}{$u_{1}$}, \colorbox{bgreen}{$\Sigma_{1}$})   \\
                                                                          & = 0.00891                                                                                           \\
                      p(\textcolor{cblue}{x_2}|\colorbox{byellow}{$c=2$}) & = \mathcal{N}(\textcolor{cblue}{x_2}|\colorbox{byellow}{$u_{2}$}, \colorbox{byellow}{$\Sigma_{2}$}) \\
                                                                          & = 0.04827
                  \end{aligned}
              $$

              \switchcolumn

              $$
                  \begin{aligned}
                      p(\textcolor{corange}{x_3}|\colorbox{bgreen}{$c=1$})  & = \mathcal{N}(\textcolor{corange}{x_3}|\colorbox{bgreen}{$u_{1}$}, \colorbox{bgreen}{$\Sigma_{1}$})   \\
                                                                            & = 0.03380                                                                                             \\
                      p(\textcolor{corange}{x_3}|\colorbox{byellow}{$c=2$}) & = \mathcal{N}(\textcolor{corange}{x_3}|\colorbox{byellow}{$u_{2}$}, \colorbox{byellow}{$\Sigma_{2}$}) \\
                                                                            & = 0.06197
                  \end{aligned}
              $$

          \end{paracol}

          \begin{center}
              Then, we multiply it by $p(c_k)$, as per \eqref{ex1-joint}, yielding
              the joint probability for each cluster,
          \end{center}

          \begin{paracol}{3}
              \begin{scriptsize}
                  $$
                      \begin{aligned}
                          p(\textcolor{cred}{x_1},\colorbox{bgreen}{$c=1$})  & = p(\textcolor{cred}{x_1}|\colorbox{bgreen}{$c=1$})p(\colorbox{bgreen}{$c=1$})   \\
                                                                             & = 0.06584 \times 0.5                                                             \\
                                                                             & = 0.03292                                                                        \\
                          p(\textcolor{cred}{x_1},\colorbox{byellow}{$c=2$}) & = p(\textcolor{cred}{x_1}|\colorbox{byellow}{$c=2$})p(\colorbox{byellow}{$c=2$}) \\
                                                                             & = 0.02280 \times 0.5                                                             \\
                                                                             & = 0.01140
                      \end{aligned}
                  $$
              \end{scriptsize}

              \switchcolumn

              \begin{scriptsize}
                  $$
                      \begin{aligned}
                          p(\textcolor{cblue}{x_2},\colorbox{bgreen}{$c=1$})  & = p(\textcolor{cblue}{x_2}|\colorbox{bgreen}{$c=1$})p(\colorbox{bgreen}{$c=1$})   \\
                                                                              & = 0.00891 \times 0.5                                                              \\
                                                                              & = 0.00446                                                                         \\
                          p(\textcolor{cblue}{x_2},\colorbox{byellow}{$c=2$}) & = p(\textcolor{cblue}{x_2}|\colorbox{byellow}{$c=2$})p(\colorbox{byellow}{$c=2$}) \\
                                                                              & = 0.04827 \times 0.5                                                              \\
                                                                              & = 0.02413
                      \end{aligned}
                  $$
              \end{scriptsize}

              \switchcolumn

              \begin{scriptsize}
                  $$
                      \begin{aligned}
                          p(\textcolor{corange}{x_3},\colorbox{bgreen}{$c=1$})  & = p(\textcolor{corange}{x_3}|\colorbox{bgreen}{$c=1$})p(\colorbox{bgreen}{$c=1$})   \\
                                                                                & = 0.03380 \times 0.5                                                                \\
                                                                                & = 0.01690                                                                           \\
                          p(\textcolor{corange}{x_3},\colorbox{byellow}{$c=2$}) & = p(\textcolor{corange}{x_3}|\colorbox{byellow}{$c=2$})p(\colorbox{byellow}{$c=2$}) \\
                                                                                & = 0.06197 \times 0.5                                                                \\
                                                                                & = 0.03099
                      \end{aligned}
                  $$
              \end{scriptsize}

          \end{paracol}

          \begin{center}
              We can now calculate the value of $p(x_i)$, given by \eqref{ex1-sum-joint},
          \end{center}

          \vspace*{0.5cm}

          \begin{paracol}{3}
              $$
                  \begin{aligned}
                      p(\textcolor{cred}{x_1}) & = \sum_k p(\textcolor{cred}{x_1}, c_k) \\
                                               & = 0.03292 + 0.01140                    \\
                                               & = 0.04432
                  \end{aligned}
              $$

              \switchcolumn

              $$
                  \begin{aligned}
                      p(\textcolor{cblue}{x_2}) & = \sum_k p(\textcolor{cblue}{x_2}, c_k) \\
                                                & = 0.00446 + 0.02413                     \\
                                                & = 0.02859
                  \end{aligned}
              $$

              \switchcolumn

              $$
                  \begin{aligned}
                      p(\textcolor{corange}{x_3}) & = \sum_k p(\textcolor{corange}{x_3}, c_k) \\
                                                  & = 0.01690 + 0.03099                       \\
                                                  & = 0.04789
                  \end{aligned}
              $$

          \end{paracol}

          \begin{center}
              Finally, we can calculate the posterior probability, using \eqref{ex1-posterior}, for each cluster,
          \end{center}

          \begin{paracol}{3}
              $$
                  \begin{aligned}
                      p(\colorbox{bgreen}{$c=1$}|\textcolor{cred}{x_1})  & = \frac{p(\textcolor{cred}{x_1},\colorbox{bgreen}{$c=1$})}{p(\textcolor{cred}{x_1})}  \\
                                                                         & = \frac{0.03292}{0.04432}                                                             \\
                                                                         & = 0.74279                                                                             \\
                      p(\colorbox{byellow}{$c=2$}|\textcolor{cred}{x_1}) & = \frac{p(\textcolor{cred}{x_1},\colorbox{byellow}{$c=2$})}{p(\textcolor{cred}{x_1})} \\
                                                                         & = \frac{0.01140}{0.04432}                                                             \\
                                                                         & = 0.25721
                  \end{aligned}
              $$

              \switchcolumn

              $$
                  \begin{aligned}
                      p(\colorbox{bgreen}{$c=1$}|\textcolor{cblue}{x_2})  & = \frac{p(\textcolor{cblue}{x_2},\colorbox{bgreen}{$c=1$})}{p(\textcolor{cblue}{x_2})}  \\
                                                                          & = \frac{0.00446}{0.02859}                                                               \\
                                                                          & = 0.15584                                                                               \\
                      p(\colorbox{byellow}{$c=2$}|\textcolor{cblue}{x_2}) & = \frac{p(\textcolor{cblue}{x_2},\colorbox{byellow}{$c=2$})}{p(\textcolor{cblue}{x_2})} \\
                                                                          & = \frac{0.02413}{0.02859}                                                               \\
                                                                          & = 0.84416
                  \end{aligned}
              $$

              \switchcolumn

              $$
                  \begin{aligned}
                      p(\colorbox{bgreen}{$c=1$}|\textcolor{corange}{x_3})  & = \frac{p(\textcolor{corange}{x_3},\colorbox{bgreen}{$c=1$})}{p(\textcolor{corange}{x_3})}  \\
                                                                            & = \frac{0.01690}{0.04789}                                                                   \\
                                                                            & = 0.35294                                                                                   \\
                      p(\colorbox{byellow}{$c=2$}|\textcolor{corange}{x_3}) & = \frac{p(\textcolor{corange}{x_3},\colorbox{byellow}{$c=2$})}{p(\textcolor{corange}{x_3})} \\
                                                                            & = \frac{0.03099}{0.04789}                                                                   \\
                                                                            & = 0.64706
                  \end{aligned}
              $$

          \end{paracol}

          \vspace*{0.5cm}

          \textbf{Maximization (M-step)}

          In this step we will update the clusters' parameters, that is, the components
          of the mutlvariate Gaussian mixture.

          For each cluster, $c_k$, we will calculate the following,

          \begin{equation}\label{ex1-nk}
              N_k = \sum_i \gamma_{ki}
          \end{equation}
          \begin{equation}\label{ex1-new-uk}
              u_k' = \frac{1}{N_k} \sum_{i} \gamma_{ki} \cdot x_i
          \end{equation}
          \begin{equation}\label{ex1-new-sigmak}
              \varSigma_k' = \frac{1}{N_k} \sum_{i} \gamma_{ki} \cdot \left(x_i - u_k\right) \cdot (x_i - u_k)^T
          \end{equation}

          Considering $N = \sum_k N_k$, we can also update the priors,

          \begin{equation}\label{ex1-new-pik}
              \pi_k' = \frac{N_k}{N}
          \end{equation}

          We can now update the values for both clusters.

          \vspace*{1.0cm}

          \begin{paracol}{2}
              \begin{center}
                  \textbf{\colorbox{bgreen}{Cluster 1}}
              \end{center}

              \switchcolumn

              \begin{center}
                  \textbf{\colorbox{byellow}{Cluster 2}}
              \end{center}
          \end{paracol}

          \begin{center}
              We start by calculating the sum of the weights of each observation,
              for each cluster, $N_k$, using \eqref{ex1-nk},
          \end{center}

          \begin{paracol}{2}
              $$
                  \begin{aligned}
                      N_1 & = \sum_{i} \gamma_{1, i}                                                                                  \\
                          & = \textcolor{cred}{\gamma_{1, 1}} + \textcolor{cblue}{\gamma_{1, 2}} + \textcolor{corange}{\gamma_{1, 3}} \\
                          & = \textcolor{cred}{0.74279} + \textcolor{cblue}{0.15584} + \textcolor{corange}{0.35294}                   \\
                          & = 1.25157
                  \end{aligned}
              $$

              \switchcolumn

              $$
                  \begin{aligned}
                      N_2 & = \sum_{i} \gamma_{2, i}                                                                                  \\
                          & = \textcolor{cred}{\gamma_{2, 1}} + \textcolor{cblue}{\gamma_{2, 2}} + \textcolor{corange}{\gamma_{2, 3}} \\
                          & = \textcolor{cred}{0.25721} + \textcolor{cblue}{0.84416} + \textcolor{corange}{0.64706}                   \\
                          & = 1.74843
                  \end{aligned}
              $$
          \end{paracol}

          \begin{center}
              We can use this to calculate $N$, the sum of $N_1$ and $N_2$,
          \end{center}
          $$
              N = N_1 + N_2 = 1.25157 + 1.74843 = 3.0
          $$

          \begin{center}
              We can now update the values of $u_k$, $\varSigma_k$ and $\pi_k$ for
              each of the clusters, using \eqref{ex1-new-uk}, \eqref{ex1-new-sigmak} and
              \eqref{ex1-new-pik} respectively.
          \end{center}


          \begin{paracol}{2}
              \begin{small}
                  $$
                      \begin{aligned}
                          u_1' & = \frac{1}{N_1} \sum_{i} \gamma_{1, i} \cdot x_i  \\
                               & = \frac{\textcolor{cblue}{\gamma_{1,1} \cdot x_1}
                              + \textcolor{cred}{\gamma_{1,2} \cdot x_2}
                          + \textcolor{corange}{\gamma_{1,3} \cdot x_3}}{N_1}      \\
                               & = \frac{\textcolor{cblue}{0.74279 \cdot
                                  \begin{pmatrix}
                                      1 \\
                                      2
                                  \end{pmatrix}
                              } + \textcolor{cred}{0.15584 \cdot
                                  \begin{pmatrix}
                                      -1 \\
                                      1
                                  \end{pmatrix}
                              } + \textcolor{corange}{0.35294 \cdot
                                  \begin{pmatrix}
                                      1 \\
                                      0
                                  \end{pmatrix}
                          }}{1.25157}                                              \\
                               & = \begin{pmatrix}
                                       0.75096 \\
                                       1.31149
                                   \end{pmatrix}
                      \end{aligned}
                  $$
              \end{small}
              \begin{footnotesize}
                  $$
                      \begin{aligned}
                          \varSigma_1' & = \frac{1}{N_1} \sum_{i} \gamma_{1, i} \cdot
                          \left(x_i - \colorbox{bgreen}{$u_1$}\right) \cdot \left(x_i - \colorbox{bgreen}{$u_1$}\right)^T                   \\
                                       & = \frac{1}{N_1} \times \Big( \textcolor{cblue}{\gamma_{1,1} \cdot (x_1 - u_1) \cdot (x_1 - u_1)^T} \\
                                       & \quad + \textcolor{cred}{\gamma_{1,2} \cdot (x_2 - u_1) \cdot (x_2 - u_1)^T}                       \\
                                       & \quad + \textcolor{corange}{\gamma_{1,3} \cdot (x_3 - u_1) \cdot (x_3 - u_1)^T}\Big)               \\
                                       & = \frac{1}{1.25157} \times \Bigg( \textcolor{cblue}{0.74279 \cdot \left(
                                  \begin{pmatrix}
                                      1 \\
                                      2
                                  \end{pmatrix}
                                  -
                                  \begin{pmatrix}
                                      2 \\
                                      2
                                  \end{pmatrix}
                                  \right) \cdot \left(
                                  \begin{pmatrix}
                                      1 \\
                                      2
                                  \end{pmatrix}
                                  -
                                  \begin{pmatrix}
                                      2 \\
                                      2
                                  \end{pmatrix}
                          \right)^T}                                                                                                        \\
                                       & \quad + \textcolor{cred}{0.15584 \cdot \left(
                                  \begin{pmatrix}
                                      -1 \\
                                      1
                                  \end{pmatrix}
                                  -
                                  \begin{pmatrix}
                                      2 \\
                                      2
                                  \end{pmatrix}
                                  \right) \cdot \left(
                                  \begin{pmatrix}
                                      -1 \\
                                      1
                                  \end{pmatrix}
                                  -
                                  \begin{pmatrix}
                                      2 \\
                                      2
                                  \end{pmatrix}
                          \right)^T}                                                                                                        \\
                                       & \quad + \textcolor{corange}{0.35294 \cdot \left(
                                  \begin{pmatrix}
                                      1 \\
                                      0
                                  \end{pmatrix}
                                  -
                                  \begin{pmatrix}
                                      2 \\
                                      2
                                  \end{pmatrix}
                                  \right) \cdot \left(
                                  \begin{pmatrix}
                                      1 \\
                                      0
                                  \end{pmatrix}
                                  -
                                  \begin{pmatrix}
                                      2 \\
                                      2
                                  \end{pmatrix}
                          \right)^T}\Bigg)                                                                                                  \\
                                       & = \begin{pmatrix}{}
                                               0.43605 & 0.07757 \\
                                               0.07757 & 0.77846
                                           \end{pmatrix}
                      \end{aligned}
                  $$
              \end{footnotesize}
              $$
                  \pi_1' = \frac{N_1}{N} = \frac{1.25157}{3.0} = 0.41719
              $$

              \switchcolumn

              \begin{small}
                  $$
                      \begin{aligned}
                          u_2' & = \frac{1}{N_2} \sum_{i} \gamma_{2, i} \cdot x_i  \\
                               & = \frac{\textcolor{cblue}{\gamma_{2,1} \cdot x_1}
                              + \textcolor{cred}{\gamma_{2,2} \cdot x_2}
                          + \textcolor{corange}{\gamma_{2,3} \cdot x_3}}{N_2}      \\
                               & = \frac{\textcolor{cblue}{0.25721 \cdot
                                  \begin{pmatrix}
                                      1 \\
                                      2
                                  \end{pmatrix}
                              } + \textcolor{cred}{0.84416 \cdot
                                  \begin{pmatrix}
                                      -1 \\
                                      1
                                  \end{pmatrix}
                              } + \textcolor{corange}{0.64706 \cdot
                                  \begin{pmatrix}
                                      1 \\
                                      0
                                  \end{pmatrix}
                          }}{1.74843}                                              \\
                               & = \begin{pmatrix}
                                       0.03438 \\
                                       0.77703
                                   \end{pmatrix}
                      \end{aligned}
                  $$
              \end{small}
              \begin{footnotesize}
                  $$
                      \begin{aligned}
                          \varSigma_2' & = \frac{1}{N_2} \sum_{i} \gamma_{2, i} \cdot
                          \left(x_i - \colorbox{byellow}{$u_2$}\right) \cdot \left(x_i - \colorbox{byellow}{$u_2$}\right)^T                 \\
                                       & = \frac{1}{N_2} \times \Big( \textcolor{cblue}{\gamma_{2,1} \cdot (x_1 - u_2) \cdot (x_1 - u_2)^T} \\
                                       & \quad + \textcolor{cred}{\gamma_{2,2} \cdot (x_2 - u_2) \cdot (x_2 - u_2)^T}                       \\
                                       & \quad + \textcolor{corange}{\gamma_{2,3} \cdot (x_3 - u_2) \cdot (x_3 - u_2)^T}\Big)               \\
                                       & = \frac{1}{1.74843} \times \Bigg( \textcolor{cblue}{0.25721 \cdot \left(
                                  \begin{pmatrix}
                                      1 \\
                                      2
                                  \end{pmatrix}
                                  -
                                  \begin{pmatrix}
                                      0 \\
                                      0
                                  \end{pmatrix}
                                  \right) \cdot \left(
                                  \begin{pmatrix}
                                      1 \\
                                      2
                                  \end{pmatrix}
                                  -
                                  \begin{pmatrix}
                                      0 \\
                                      0
                                  \end{pmatrix}
                          \right)^T}                                                                                                        \\
                                       & \quad + \textcolor{cred}{0.84416 \cdot \left(
                                  \begin{pmatrix}
                                      -1 \\
                                      1
                                  \end{pmatrix}
                                  -
                                  \begin{pmatrix}
                                      0 \\
                                      0
                                  \end{pmatrix}
                                  \right) \cdot \left(
                                  \begin{pmatrix}
                                      -1 \\
                                      1
                                  \end{pmatrix}
                                  -
                                  \begin{pmatrix}
                                      0 \\
                                      0
                                  \end{pmatrix}
                          \right)^T}                                                                                                        \\
                                       & \quad + \textcolor{corange}{0.64706 \cdot \left(
                                  \begin{pmatrix}
                                      1 \\
                                      0
                                  \end{pmatrix}
                                  -
                                  \begin{pmatrix}
                                      0 \\
                                      0
                                  \end{pmatrix}
                                  \right) \cdot \left(
                                  \begin{pmatrix}
                                      1 \\
                                      0
                                  \end{pmatrix}
                                  -
                                  \begin{pmatrix}
                                      0 \\
                                      0
                                  \end{pmatrix}
                          \right)^T} \Bigg)                                                                                                 \\
                                       & = \begin{pmatrix}{}
                                               0.99882  & -0.21531 \\
                                               -0.21531 & 0.46748
                                           \end{pmatrix}
                      \end{aligned}
                  $$
              \end{footnotesize}
              $$
                  \pi_2' = \frac{N_2}{N} = \frac{1.74843}{3.0} = 0.58281
              $$

          \end{paracol}

          \vspace*{0.5cm}

          \textbf{Evaluate the log likelihood}

          Since we are only performing one epoch of the EM clustering algorithm,
          we can skip this step.

          \vspace*{1.0cm}

          \textbf{Conclusion}

          After performing one epoch of the EM clustering algorithm, we end
          up with the following parameters for each cluster.

          \begin{center}
              \captionsetup{type=table}
              \begin{tabular}{c|ccc}
                  Cluster                       & $u$ & $\varSigma$ & $\pi$ \\
                  \hline
                  \colorbox{bgreen}{Cluster 1}  &
                  $\begin{pmatrix}
                           0.75096 \\
                           1.31149
                       \end{pmatrix}$              &
                  $\begin{pmatrix}{}
                           0.43605 & 0.07757 \\
                           0.07757 & 0.77846
                       \end{pmatrix}$          &
                  0.41719                                                   \\
                  \colorbox{byellow}{Cluster 2} &
                  $\begin{pmatrix}
                           0.03438 \\
                           0.77703
                       \end{pmatrix}$              &
                  $\begin{pmatrix}{}
                           0.99882  & -0.21531 \\
                           -0.21531 & 0.46748
                       \end{pmatrix}$        &
                  0.58281                                                   \\
              \end{tabular}
              \captionof{table}{Parameters for the 2 clusters in the multivariate Gaussian mixture after one epoch of the EM clustering algorithm}
              \label{ex1-final-params-table}
          \end{center}

    \item {\color{questioncolor}\bfseries
          Given the updated parameters computed in the previous question:
          }\\
          \vspace{-1.0em}

          \begin{enumerate}
              \item {\color{questioncolor}\bfseries
                    perform a hard assignment of observations to clusters under a
                    MAP assumption.
                    }\\
                    \vspace{0.5em}

                    As we've done in exercise 1, we can use \eqref{ex1-posterior} to calculate the
                    posterior probabilities of each cluster for each observation, in order to assign
                    them to a cluster.

                    \vspace*{0.5cm}


          \end{enumerate}

          \begin{paracol}{3}
              \begin{center}
                  \textbf{Observation: \textcolor{cred}{$x_1 = \begin{pmatrix}
                                  1 \\
                                  2
                              \end{pmatrix}$}}
              \end{center}

              \switchcolumn

              \begin{center}
                  \textbf{Observation: \textcolor{cblue}{$x_2 = \begin{pmatrix}
                                  -1 \\
                                  1
                              \end{pmatrix}$}}
              \end{center}

              \switchcolumn

              \begin{center}
                  \textbf{Observation: \textcolor{corange}{$x_3 = \begin{pmatrix}
                                  1 \\
                                  0
                              \end{pmatrix}$}}
              \end{center}

          \end{paracol}

          \begin{center}
              We'll start by calculating the likelihood, using \eqref{ex1-likelihood},
              for each cluster,
          \end{center}

          \begin{paracol}{3}
              $$
                  \begin{aligned}
                      p(\textcolor{cred}{x_1}|\colorbox{bgreen}{$c=1$})  & = \mathcal{N}(\textcolor{cred}{x_1}|\colorbox{bgreen}{$u_{1}'$}, \colorbox{bgreen}{$\Sigma_{1}'$})   \\
                                                                         & = 0.19570                                                                                            \\
                      p(\textcolor{cred}{x_1}|\colorbox{byellow}{$c=2$}) & = \mathcal{N}(\textcolor{cred}{x_1}|\colorbox{byellow}{$u_{2}'$}, \colorbox{byellow}{$\Sigma_{2}'$}) \\
                                                                         & = 0.01352
                  \end{aligned}
              $$

              \switchcolumn

              $$
                  \begin{aligned}
                      p(\textcolor{cblue}{x_2}|\colorbox{bgreen}{$c=1$})  & = \mathcal{N}(\textcolor{cblue}{x_2}|\colorbox{bgreen}{$u_{1}'$}, \colorbox{bgreen}{$\Sigma_{1}'$})   \\
                                                                          & = 0.00820                                                                                             \\
                      p(\textcolor{cblue}{x_2}|\colorbox{byellow}{$c=2$}) & = \mathcal{N}(\textcolor{cblue}{x_2}|\colorbox{byellow}{$u_{2}'$}, \colorbox{byellow}{$\Sigma_{2}'$}) \\
                                                                          & = 0.14365
                  \end{aligned}
              $$

              \switchcolumn

              $$
                  \begin{aligned}
                      p(\textcolor{corange}{x_3}|\colorbox{bgreen}{$c=1$})  & = \mathcal{N}(\textcolor{corange}{x_3}|\colorbox{bgreen}{$u_{1}'$}, \colorbox{bgreen}{$\Sigma_{1}'$})   \\
                                                                            & = 0.07717                                                                                               \\
                      p(\textcolor{corange}{x_3}|\colorbox{byellow}{$c=2$}) & = \mathcal{N}(\textcolor{corange}{x_3}|\colorbox{byellow}{$u_{2}'$}, \colorbox{byellow}{$\Sigma_{2}'$}) \\
                                                                            & = 0.10478
                  \end{aligned}
              $$

          \end{paracol}

          \begin{center}
              Then, we multiply it by $p(c_k)$, as per \eqref{ex1-joint}, yielding
              the joint probability for each cluster,
          \end{center}

          \begin{paracol}{3}
              \begin{scriptsize}
                  $$
                      \begin{aligned}
                          p(\textcolor{cred}{x_1},\colorbox{bgreen}{$c=1$})  & = p(\textcolor{cred}{x_1}|\colorbox{bgreen}{$c=1$})p(\colorbox{bgreen}{$c=1$})   \\
                                                                             & = 0.19570 \times 0.41719                                                         \\
                                                                             & = 0.08164                                                                        \\
                          p(\textcolor{cred}{x_1},\colorbox{byellow}{$c=2$}) & = p(\textcolor{cred}{x_1}|\colorbox{byellow}{$c=2$})p(\colorbox{byellow}{$c=2$}) \\
                                                                             & = 0.01352 \times 0.58281                                                         \\
                                                                             & = 0.00788
                      \end{aligned}
                  $$
              \end{scriptsize}

              \switchcolumn

              \begin{scriptsize}
                  $$
                      \begin{aligned}
                          p(\textcolor{cblue}{x_2},\colorbox{bgreen}{$c=1$})  & = p(\textcolor{cblue}{x_2}|\colorbox{bgreen}{$c=1$})p(\colorbox{bgreen}{$c=1$})   \\
                                                                              & = 0.00820 \times 0.41719                                                          \\
                                                                              & = 0.00342                                                                         \\
                          p(\textcolor{cblue}{x_2},\colorbox{byellow}{$c=2$}) & = p(\textcolor{cblue}{x_2}|\colorbox{byellow}{$c=2$})p(\colorbox{byellow}{$c=2$}) \\
                                                                              & = 0.14365 \times 0.58281                                                          \\
                                                                              & = 0.08372
                      \end{aligned}
                  $$
              \end{scriptsize}

              \switchcolumn

              \begin{scriptsize}
                  $$
                      \begin{aligned}
                          p(\textcolor{corange}{x_3},\colorbox{bgreen}{$c=1$})  & = p(\textcolor{corange}{x_3}|\colorbox{bgreen}{$c=1$})p(\colorbox{bgreen}{$c=1$})   \\
                                                                                & = 0.07717 \times 0.41719                                                            \\
                                                                                & = 0.03219                                                                           \\
                          p(\textcolor{corange}{x_3},\colorbox{byellow}{$c=2$}) & = p(\textcolor{corange}{x_3}|\colorbox{byellow}{$c=2$})p(\colorbox{byellow}{$c=2$}) \\
                                                                                & = 0.10478 \times 0.58281                                                            \\
                                                                                & = 0.06107
                      \end{aligned}
                  $$
              \end{scriptsize}

          \end{paracol}

          \begin{center}
              We can now calculate the value of $p(x_i)$, given by \eqref{ex1-sum-joint},
          \end{center}

          \begin{paracol}{3}
              $$
                  \begin{aligned}
                      p(\textcolor{cred}{x_1}) & = \sum_k p(\textcolor{cred}{x_1}, c_k) \\
                                               & = 0.08164 + 0.00788                    \\
                                               & = 0.08952
                  \end{aligned}
              $$

              \switchcolumn

              $$
                  \begin{aligned}
                      p(\textcolor{cblue}{x_2}) & = \sum_k p(\textcolor{cblue}{x_2}, c_k) \\
                                                & = 0.00342 + 0.08372                     \\
                                                & = 0.08714
                  \end{aligned}
              $$

              \switchcolumn

              $$
                  \begin{aligned}
                      p(\textcolor{corange}{x_3}) & = \sum_k p(\textcolor{corange}{x_3}, c_k) \\
                                                  & = 0.03219 + 0.06107                       \\
                                                  & = 0.09326
                  \end{aligned}
              $$

          \end{paracol}

          \begin{center}
              Finally, we can calculate the posterior probability, using \eqref{ex1-posterior}, for each cluster,
              and then assign each observation to a cluster, under a MAP assumption,
          \end{center}

          \begin{paracol}{3}
              $$
                  \begin{aligned}
                      p(\colorbox{bgreen}{$c=1$}|\textcolor{cred}{x_1})  & = \frac{p(\textcolor{cred}{x_1},\colorbox{bgreen}{$c=1$})}{p(\textcolor{cred}{x_1})}  \\
                                                                         & = \frac{0.08164}{0.08952}                                                             \\
                                                                         & = 0.91198                                                                             \\
                      p(\colorbox{byellow}{$c=2$}|\textcolor{cred}{x_1}) & = \frac{p(\textcolor{cred}{x_1},\colorbox{byellow}{$c=2$})}{p(\textcolor{cred}{x_1})} \\
                                                                         & = \frac{0.00788}{0.08952}                                                             \\
                                                                         & = 0.08802
                  \end{aligned}
              $$

              Therefore, \textcolor{cred}{$x_1$} is assigned to\\
              \colorbox{bgreen}{Cluster 1}.

              \switchcolumn

              $$
                  \begin{aligned}
                      p(\colorbox{bgreen}{$c=1$}|\textcolor{cblue}{x_2})  & = \frac{p(\textcolor{cblue}{x_2},\colorbox{bgreen}{$c=1$})}{p(\textcolor{cblue}{x_2})}  \\
                                                                          & = \frac{0.00342}{0.08714}                                                               \\
                                                                          & = 0.03924                                                                               \\
                      p(\colorbox{byellow}{$c=2$}|\textcolor{cblue}{x_2}) & = \frac{p(\textcolor{cblue}{x_2},\colorbox{byellow}{$c=2$})}{p(\textcolor{cblue}{x_2})} \\
                                                                          & = \frac{0.08372}{0.08714}                                                               \\
                                                                          & = 0.96076
                  \end{aligned}
              $$

              Therefore, \textcolor{cblue}{$x_2$} is assigned to\\
              \colorbox{byellow}{Cluster 2}.

              \switchcolumn

              $$
                  \begin{aligned}
                      p(\colorbox{bgreen}{$c=1$}|\textcolor{corange}{x_3})  & = \frac{p(\textcolor{corange}{x_3},\colorbox{bgreen}{$c=1$})}{p(\textcolor{corange}{x_3})}  \\
                                                                            & = \frac{0.03219}{0.09326}                                                                   \\
                                                                            & = 0.34519                                                                                   \\
                      p(\colorbox{byellow}{$c=2$}|\textcolor{corange}{x_3}) & = \frac{p(\textcolor{corange}{x_3},\colorbox{byellow}{$c=2$})}{p(\textcolor{corange}{x_3})} \\
                                                                            & = \frac{0.06107}{0.09326}                                                                   \\
                                                                            & = 0.65481
                  \end{aligned}
              $$

              Therefore, \textcolor{corange}{$x_3$} is assigned to\\
              \colorbox{byellow}{Cluster 2}.

          \end{paracol}

          \vspace*{0.5cm}

          \begin{enumerate}[resume]
              \item {\color{questioncolor}\bfseries
                    compute the silhouette of the larger cluster using the
                    Euclidean distance.
                    }\\
                    \vspace{0.5em}

                    The larger cluster is \colorbox{byellow}{Cluster 2}, since is has 2 observations, versus just one
                    in the other cluster.

                    The silhouette of each point, $s(x_i)$, is given by,
                    \begin{equation}\label{ex2b-ai-bi}
                        a_i = \frac{1}{n_k - 1} \sum_{j \in c_k \backslash \{i\}} d(x_i, x_j)
                        \quad,\quad
                        b_i = \min_{k' \ne k} \frac{1}{n_k'} \sum_{j \in c_{k'}} d(x_i, x_j)
                    \end{equation}
                    \begin{equation}\label{ex2b-s-xi}
                        s(x_i) = \begin{cases}
                            1-\frac{a_i}{b_i}   & , \text{if}~a_i < b_i    \\
                            \frac{b_i}{a_i} - 1 & , \text{if}~a_i \geq b_i
                        \end{cases}
                        =\frac{b_i - a_i}{\max(a_i,b_i)}
                    \end{equation}
                    where $d(x_i, x_j)$ is the Euclidean distance between $x_i$ and $x_j$.

                    For the sake of completeness, the Euclidean distance between two points in a $d$-dimensional
                    space is given by

                    $$
                        d(x_i, x_j) = ||x_i - x_m||_d = \sqrt{\sum_{m=1}^{d} \left(x_{i_m} - x_{j_m}\right)^2}
                    $$

                    It is relevant to note that $a_i$ is the mean of the distances of observation $x_i$ to other
                    observations in its cluster, while $b_i$ is the mean of the distances of
                    that same observation to the observations in another cluster (the cluster
                    that yeilds the lowest value).

                    The silhouette of a cluster, $s(c_k)$ is the mean of the silhouette of its observations,
                    \begin{equation}\label{ex2b-s-ck}
                        s(c_k) = \frac{1}{n_k} \sum_{i \in c_k} s_i
                    \end{equation}

                    Observations \textcolor{cblue}{$x_2$} and \textcolor{corange}{$x_3$}
                    belong to \colorbox{byellow}{Cluster 2}, as per the previous
                    question, therefore we'll calculate their silhouettes:

                    \vspace*{0.5cm}

                    \begin{paracol}{2}
                        \begin{center}
                            \textbf{Observation: \textcolor{cblue}{$x_2 = \begin{pmatrix}
                                            -1 \\
                                            1
                                        \end{pmatrix}$}}
                        \end{center}

                        \switchcolumn

                        \begin{center}
                            \textbf{Observation: \textcolor{corange}{$x_3 = \begin{pmatrix}
                                            1 \\
                                            0
                                        \end{pmatrix}$}}
                        \end{center}
                    \end{paracol}

                    \begin{center}
                        We can calculate the distances to other observations in
                        the same cluster (\colorbox{byellow}{Cluster 2}), $a_i$, using
                        \eqref{ex2b-ai-bi}.
                    \end{center}

                    \begin{paracol}{2}

                        $$
                            \begin{aligned}
                                \textcolor{cblue}{a_2} & = \frac{||\textcolor{cblue}{x_2} - \textcolor{corange}{x_3}||_2}{2 - 1} \\
                                                       & = ||\textcolor{cblue}{x_2} - \textcolor{corange}{x_3}||_2               \\
                                                       & = \sqrt{\left(-1 -1\right)^2 + \left(1 - 0\right)^2}                    \\
                                                       & = 2.23607
                            \end{aligned}
                        $$

                        \switchcolumn

                        $$
                            \begin{aligned}
                                \textcolor{corange}{a_3} & = \frac{||\textcolor{corange}{x_3} - \textcolor{cblue}{x_2}||_2}{2 - 1} \\
                                                         & = ||\textcolor{corange}{x_3} - \textcolor{cblue}{x_2}||_2               \\
                                                         & = \sqrt{\left(1 -\left(-1\right)\right)^2 + \left(0 - 1\right)^2}       \\
                                                         & = 2.23607
                            \end{aligned}
                        $$

                    \end{paracol}

                    \begin{center}
                        We can calculate the distances to other observations in
                        the other cluster (\colorbox{bgreen}{Cluster 1}), $b_i$, using
                        \eqref{ex2b-ai-bi}.
                    \end{center}

                    \begin{paracol}{2}

                        $$
                            \begin{aligned}
                                \textcolor{cblue}{b_2} & = \frac{||\textcolor{cblue}{x_2} - \textcolor{cred}{x_1}||_2}{1} \\
                                                       & = ||\textcolor{cblue}{x_2} - \textcolor{cred}{x_1}||_2           \\
                                                       & = \sqrt{\left(-1 -1\right)^2 + \left(1 - 2\right)^2}             \\
                                                       & = 2.23607
                            \end{aligned}
                        $$

                        \switchcolumn

                        $$
                            \begin{aligned}
                                \textcolor{corange}{b_3} & = \frac{||\textcolor{corange}{x_3} - \textcolor{cred}{x_1}||_2}{1} \\
                                                         & = ||\textcolor{corange}{x_3} - \textcolor{cred}{x_1}||_2           \\
                                                         & = \sqrt{\left(1 - 1\right)^2 + \left(0 - 2\right)^2}               \\
                                                         & = 2.0
                            \end{aligned}
                        $$

                    \end{paracol}

                    \begin{center}
                        We can now calculate the silhouette of each observation, $s(x_i)$,
                        using \eqref{ex2b-s-xi}.
                    \end{center}

                    \begin{paracol}{2}

                        $$
                            \begin{aligned}
                                \textcolor{cblue}{s(x_2)} & = \frac{b_2 - a_2}{\max(a_2, b_2)}  \\
                                                          & = \frac{2.23607 - 2.23607}{2.23607} \\
                                                          & = 0
                            \end{aligned}
                        $$

                        \switchcolumn

                        $$
                            \begin{aligned}
                                \textcolor{corange}{s(x_3)} & = \frac{b_3 - a_3}{\max(a_3, b_3)} \\
                                                            & = \frac{2.0 - 2.23607}{2.23607}    \\
                                                            & = -0.10557
                            \end{aligned}
                        $$

                    \end{paracol}

                    Finally, we're able to calculate the silhouette of \colorbox{byellow}{Cluster 2},
                    $s(c_2)$, using \eqref{ex2b-s-ck},

                    $$
                        \colorbox{byellow}{$s(c_k)$} = \frac{\textcolor{cblue}{s(x_2)} + \textcolor{corange}{s(x_3)}}{2}
                        = \frac{0 - 0.10557}{2} = -0.05279
                    $$

          \end{enumerate}
\end{enumerate}

\pagebreak

\begin{center}
    \large{\textbf{Part II}: Programming and critical analysis}
\end{center}

{\color{questioncolor}\bfseries
\noindent
Recall the \texttt{pd\_speech.arff} dataset from earlier homeworks, centered on
the Parkinson diagnosis from speech features.
For the following exercises, normalize the data using \texttt{sklearn}'s
MinMaxScaler.
}

\begin{enumerate}[leftmargin=\labelsep]
    \item {\color{questioncolor}\bfseries
          Using \texttt{sklearn}, apply \textit{k}-means clustering fully unsupervisedly
          (without targets) on the normalized data with $k = 3$ and three different
          seeds (using \texttt{random} $\in \{0,1,2\}$).
          Assess the silhouette and purity of the produced solutions.
          }\\
          \vspace{0.5em}

          Using \texttt{sklearn}'s \texttt{KMeans} class, we can apply a
          \textit{k}-means clustering algorithm.\\
          We opted to use Euclidean distance for the silhouette, since no other distance
          function was specified in the question prompt
          and it is \texttt{sklearn}'s default for the
          \texttt{metrics.silhouette\_score} function.\\
          For the purity score, we opted to use the \texttt{purity\_score} function
          available in the course's N5 (Clustering) Notebook.

          Therefore, for each seed, we get the following results:

          \begin{center}
              \captionsetup{type=table}
              \begin{tabular}{c|cc}
                  \texttt{random}        & Silhouette (Euclidean) & Purity  \\
                  \hline
                  \textcolor{cred}{0}    & 0.11362                & 0.76720 \\
                  \textcolor{cblue}{1}   & 0.11404                & 0.76323 \\
                  \textcolor{corange}{2} & 0.11362                & 0.76720
              \end{tabular}
              \captionof{table}{Silhouette (Euclidean) and purity scores for \texttt{random} $\in \{0,1,2\}$}
              \label{ex1p-silhouette-purity}
          \end{center}

          The respective code can be found on listing \ref{listing-ex1}.

    \item {\color{questioncolor}\bfseries
          What is causing the non-determinism?
          }\\
          \vspace{0.5em}

          To understand why \texttt{sklearn}'s \texttt{cluster.KMeans} is not deterministic,
          we must first understand how it works.
          To initialize the centroids' positions, \texttt{KMeans} uses the \texttt{k-means++}
          algorithm by default (and it is what we're using here).
          This algorithm starts by selecting a random point of the dataset as the first
          initial cluster centroid, followed by randomly selecting the remaining $k-1$ centroids,
          given that the points furthest away have a greater chance of being selected.
          In other words, this means the initial cluster centroids are selected
          \textit{"using sampling based on an empirical probability distribution of
              the points' contribution to the overall inertia"}, as per the
          \href{https://scikit-learn.org/stable/modules/generated/sklearn.cluster.KMeans.html}{documentation},
          therefore being generally distant from each other.

          The algorithm will then proceed based on these initial centroids.
          It is also worth noting that this process is repeated \texttt{n\_init} times (by default, 10),
          preserving the best output (i.e. the one that minimizes inertia).

          All of this means the algorithm is inheritly non-deterministic: the initial cluster
          centroids are selected randomly (not truly randomly, as we've seen, but some randomness is involved)
          and might converge to different positions (i.e. different local minimums) in different runs.

          We can see this happening on the chosen seeds: seeds \textcolor{cred}{0}
          and \textcolor{corange}{2} converge to the same centroid positions,
          while seed \textcolor{cblue}{1} converges to different centroid positions.

    \item {\color{questioncolor}\bfseries
          Using a scatter plot, visualize side-by-side the labeled data using labels:
          i) the original Parkinson diagnosis, and ii) the previously learned $k = 3$
          clusters (\texttt{random} = 0).
          To this end, select the two most informative features as axes and color
          observations according to their label.
          For feature selection, select the two input variables with highest
          variance on the MinMax normalized data.
          }\\
          \vspace{0.5em}

          By analysing the normalized data, we choose the features with the highest
          variance to be the ones that will be used in the scatter plot.
          The most informative features are \texttt{tqwt\_entropy\_shannon\_dec\_16}
          and \texttt{tqwt\_kurtosisValue\_dec\_34}.

          Labeling the observations yields the following scatterplots:

          \begin{figure}[H]
              \centering
              \includesvg[width=17cm]{assets/hw4-scatterplots.svg}
              \caption{Side-by-side comparison of the labeled data: on the left, the original Parkinson disease labels, on the right, the cluster labeled data}
              \label{fig:scatterplot-labels}
          \end{figure}

          The respective code can be found on listing \ref{listing-ex3}.

    \item {\color{questioncolor}\bfseries
          The fraction of variance explained by a principal component is the ratio
          between the variance of that component (i.e., its eigenvalue) and total
          variance (i.e., sum of all eigenvalues).
          How many principal components are necessary to explain more than 80\%
          of variability?\\
          Hint: explore the \texttt{DimReduction} notebook to be familiar
          with PCA in \texttt{sklearn}.
          }\\
          \vspace{0.5em}

          By reading the documentation of the
          \href{https://scikit-learn.org/stable/modules/generated/sklearn.decomposition.PCA.html}{\texttt{decomposition.PCA} class}
          in \texttt{sklearn}, at the moment of initalization of the \texttt{PCA} we can give
          the \texttt{n\_components} parameter a value between 0 and 1 (in this case 0.8),
          and set the \texttt{svd\_solver} parameter with the value \texttt{full}.
          With these parameters, the PCA will return the number of components that explain more than 80\% of the variance.


          Fitting this model to our normalized data set and consulting the parameter
          \texttt{n\_components\_} of the model, gives us the number of principal components
          as requested by the question.
          Therefore, \textbf{31} principal components are necessary to explain more than 80\% of variability.

          The respective code can be found on listing \ref{listing-ex4}.

\end{enumerate}

\pagebreak

\center\large{\textbf{Appendix}\vskip 0.3cm}

\lstinputlisting[label={listing-ex1},caption={Apply 3 k-means and assess the silhouette and purity of the produced solutions},language=Python]{assets/hw4-code-1.py}

\vspace*{1.5cm}

\lstinputlisting[label={listing-ex3},caption={Scatter plot of the the original data and the cluster (k = 3) with random = 0},language=Python]{assets/hw4-code-2.py}

\vspace*{1.5cm}

\lstinputlisting[label={listing-ex4},caption={Number of principal components that are necessary to explain more than 80\% of variablity},language=Python]{assets/hw4-code-3.py}


\end{document}

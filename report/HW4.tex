\documentclass[12pt]{article}
\usepackage[paper=letterpaper,margin=2cm]{geometry}
\usepackage{amsmath}
\usepackage{amssymb}
\usepackage{amsfonts}
\usepackage{newtxtext, newtxmath}
\usepackage{enumitem}
\usepackage{titling}
\usepackage[colorlinks=true]{hyperref}
\usepackage{multirow}
\usepackage{svg}
\usepackage{listings}
\usepackage{xcolor}
\usepackage{float}
\usepackage{graphicx}
\usepackage{subcaption}
\usepackage{paracol}

\setlength{\droptitle}{-6em}

\definecolor{codegreen}{rgb}{0,0.6,0}
\definecolor{codegray}{rgb}{0.5,0.5,0.5}
\definecolor{codepurple}{rgb}{0.58,0,0.82}
\definecolor{backcolour}{rgb}{0.95,0.95,0.92}

% TEXT COLORS
% https://coolors.co/4c6085-1588e0-0cac8c-f25f5c-b89300
\definecolor{cgray}{HTML}{4c6085}
\definecolor{cblue}{HTML}{1588e0}
\definecolor{cgreen}{HTML}{0cac8c}
\definecolor{cred}{HTML}{f25f5c}
\definecolor{cyellow}{HTML}{b89300}

% BACKGROUND COLORS
% https://coolors.co/8a9cbc-7cc0f3-66f4d8-f6908e-ffe270
\definecolor{bgray}{HTML}{8a9cbc}
\definecolor{bblue}{HTML}{7cc0f3}
\definecolor{bgreen}{HTML}{66f4d8}
\definecolor{bred}{HTML}{f6908e}
\definecolor{byellow}{HTML}{ffe270}

\definecolor{linkcolor}{HTML}{f57429}
\definecolor{questioncolor}{HTML}{444444}
\definecolor{listingbackground}{HTML}{f5f5f5}

\hypersetup{
    allcolors=linkcolor
}

\setlength{\columnseprule}{1pt}
\def\columnseprulecolor{\color{questioncolor}}

\lstdefinestyle{mystyle}{
    commentstyle=\color{codegreen},
    keywordstyle=\color{magenta},
    numberstyle=\tiny\color{codegray},
    stringstyle=\color{codepurple},
    basicstyle=\ttfamily\footnotesize,
    backgroundcolor=\color{listingbackground},
    breakatwhitespace=false,
    breaklines=true,
    captionpos=b,
    keepspaces=true,
    numbers=left,
    numbersep=5pt,
    showspaces=false,
    showstringspaces=false,
    showtabs=false,
    tabsize=2
}

\lstset{style=mystyle}

\title{\large{Aprendizagem 2022}\vskip 0.2cm Homework IV -- Group 020}
\date{}
\author{Diogo Correia (99211) \and Tomás Esteves (99341)}
\begin{document}
\maketitle
\begin{center}
    \large{\vskip -1.0cm\textbf{Part I}: Pen and paper}
\end{center}

{
\color{questioncolor}\bfseries
\noindent
Given the bivariate observations $
    \left\{
    \begin{pmatrix}
        1 \\
        2
    \end{pmatrix},
    \begin{pmatrix}
        -1 \\
        1
    \end{pmatrix},
    \begin{pmatrix}
        1 \\
        0
    \end{pmatrix}
    \right\}
$, and the multivariate Gaussian mixture
$$
    u_1 = \begin{pmatrix}
        2 \\
        2
    \end{pmatrix}
    \quad,\quad
    u_2 = \begin{pmatrix}
        0 \\
        0
    \end{pmatrix}
    \quad,\quad
    \varSigma_1 = \begin{pmatrix}
        2 & 1 \\
        1 & 2
    \end{pmatrix}
    \quad,\quad
    \varSigma_2 = \begin{pmatrix}
        2 & 0 \\
        0 & 2
    \end{pmatrix}
    \quad,\quad
    \pi_1 = 0.5
    \quad,\quad
    \pi_2 = 0.5
$$
}

\begin{enumerate}[leftmargin=\labelsep]
    \item {\color{questioncolor}\bfseries
          Perform one epoch of the EM clustering algorithm and determine the new parameters.
          Indicate all calculus step by step (you can use a computer, however disclose
          intermediary steps).
          }\\
          \vspace{0.5em}

          // TODO

    \item {\color{questioncolor}\bfseries
          Given the updated parameters computed in the previous question:
          }\\
          \vspace{-1.0em}

          \begin{enumerate}
              \item {\color{questioncolor}\bfseries
                    perform a hard assignment of observations to clusters under a
                    MAP assumption.
                    }\\
                    \vspace{0.5em}
                    // TODO
              \item {\color{questioncolor}\bfseries
                    compute the silhouette of the larger cluster using the
                    Euclidean distance.
                    }\\
                    \vspace{0.5em}
                    // TODO
          \end{enumerate}
\end{enumerate}

\pagebreak

\begin{center}
    \large{\textbf{Part II}: Programming and critical analysis}
\end{center}

{\color{questioncolor}\bfseries
\noindent
Recall the \texttt{pd\_speech.arff} dataset from earlier homeworks, centered on
the Parkinson diagnosis from speech features.
For the following exercises, normalize the data using \texttt{sklearn}'s
MinMaxScaler.
}

\begin{enumerate}[leftmargin=\labelsep]
    \item {\color{questioncolor}\bfseries
          Using \texttt{sklearn}, apply \textit{k}-means clustering fully unsupervisedly
          (without targets) on the normalized data with $k = 3$ and three different
          seeds (using \texttt{random} $\in \{0,1,2\}$).
          Assess the silhouette and purity of the produced solutions.
          }\\
          \vspace{0.5em}

          // TODO

    \item {\color{questioncolor}\bfseries
          What is causing the non-determinism?
          }\\
          \vspace{0.5em}

          // TODO

    \item {\color{questioncolor}\bfseries
          Using a scatter plot, visualize side-by-side the labeled data using labels:
          i) the original Parkinson diagnosis, and ii) the previously learned $k = 3$
          clusters (\texttt{random} = 0).
          To this end, select the two most informative features as axes and color
          observations according to their label.
          For feature selection, select the two input variables with highest
          variance on the MinMax normalized data.
          }\\
          \vspace{0.5em}

          // TODO

    \item {\color{questioncolor}\bfseries
          The fraction of variance explained by a principal component is the ratio
          between the variance of that component (i.e., its eigenvalue) and total
          variance (i.e., sum of all eigenvalues).
          How many principal components are necessary to explain more than 80\%
          of variability?\\
          Hint: explore the \texttt{DimReduction} notebook to be familiar
          with PCA in \texttt{sklearn}.
          }\\
          \vspace{0.5em}

          // TODO

\end{enumerate}

\pagebreak

\center\large{\textbf{Appendix}\vskip 0.3cm}

// TODO

\end{document}
